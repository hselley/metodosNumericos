\chapter{Solución de ecuaciones diferenciales}

Las ecuaciones diferenciales sirven para modelar problemas de ciencias e ingeniería que involucran el cambio de una 
variable con respecto a otra. En la mayor parte de ellos hay que resolver un problema de valores iniciales, es decir, 
resolver una ecuación diferencial que satisface una condición inicial dada.

En situaciones comunes de la vida real, la ecuación diferencial que modela el problema puede resultar ser demasiado 
complicada para resolverla de forma analítica, por lo que se recurre a métodos de aproximación que nos permitan conocer 
la solución numérica de la ecuación diferencial con una cota de error sin necesidad de resolverla analíticamente.

\section{Problemas bien planteados}
Para aplicar cualquiera de los métodos en este capítulo, es necesario plantear el problema con toda la información
que los métodos requieren para su aplicación. Esto se consigue a través de los problemas bien planteados.

\begin{definition}
	Se dice que el problema de valores iniciales
	\begin{align}
		\dfrac{dy}{dt} &= f(t,y) \nonumber\\
		a&\leq t\leq b \nonumber\\
		y(a)&=\alpha
		\label{eq:EDBienPlanteada}
	\end{align}
	es un \textbf{problema bien planteado} si:
	\begin{itemize}
		\item el problema tiene una solución única: $y(t)$
		\item existen constantes $\epsilon_0>0$ y $k>0$ tales que para toda $\epsilon$, con $\epsilon_0>\epsilon>0$, siempre que $\delta(t)$ sea continua con
			$|\delta (t)|<\epsilon$ para toda $t\in [a,b]$ y cuando $|\delta_0| < \epsilon$, el problema con valores iniciales
			\begin{equation}
				{dz\over dt} = f(t,z), \, a\leq t\leq b, \, z(a) = \alpha + \delta_0,
			\end{equation}
	\end{itemize}
	tiene una única solución $z(t)$ que satisface
	\begin{center}
		$|z(t) - y(t)| < k\epsilon$ para toda $t\in [a,b]$.
	\end{center}	 
\end{definition}


\section{Método de Euler}
El método de Euler es la técnica de aproximación más sencilla para resolver problemas con valores iniciales. Aunque rara vez 
se emplea en la práctica, la simplicidad de su deducción sirve para ejemplificar las técnicas con las que se desarrollan 
algunos de los métodos más avanzados, sin el álgebra tan engorrosa que acompaña a tales desarrollos.
Este método tiene por objetivo obtener una aproximación de un problema bien planteado con valores iniciales

\begin{equation}
	{dy\over dt} = f(t,y), \\ 
	a\leq t\leq b, \\ 
	y(a) = \alpha.
	\label{eq:PVIBienPlanteado}
\end{equation}

Puede que no se obtenga una aproximación continua a la solución $y(t)$; por el contrario, se generarán aproximaciones a $y$ 
en varios valores, llamados \textbf{puntos de red}, en el intervalo $[a,b]$. Una vez obtenida la solución aproximada en los 
puntos, podemos obtener por interpolación la solución aproximada en otros puntos del intervalo.

En primer lugar estipulamos que los puntos de red tienen una distribución uniforme en todo el intervalo $[a,b]$. Garantizamos 
esta condición al seleccionar un entero positivo $N$ (\textbf{numero de pasos}) y los puntos de red
\begin{align*}
	t_i &= a + ih, \\
	&\forall\, i=0,1,2,\dots,N.
\end{align*}

La distancia común entre los puntos 
\begin{align*}
	h=\dfrac{b-a}{N} = t_{i+1}-t_i
\end{align*}
recibe el nombre de \textbf{tamaño de paso}. 

Utilizaremos el teorema de Taylor para deducir el método de Euler. Supongamos que $y(t)$, la única solución de la ecuación 
(\ref{eq:PVIBienPlanteado}), tiene dos derivadas continuas en $[a,b]$, de modo que para cada $i=0,1,2,\dots,N-1$,
\[y(t_{i+1}) = y(t_i) + (t_{i+1} - t_i)y'(t_i) + \frac{(t_{i+1} - t_i)^2}{2}y''(\xi_i),\]
para algún número $\xi_i$ en $(t_i, t_{i+1})$. Como $h=t_{i+1} - t_i$, se tiene
\[y(t_{i+1}) = y(t_i) + hy'(t_i)) + \frac{h^2}{2}y''(\xi_i),\]
y como $y(t)$ satisface la ecuación diferencial \ref{eq:PVIBienPlanteado},
\begin{equation}
	y(t_{i+1}) = y(t_i) + hf(t_i, y(t_i)) + \frac{h^2}{2}y''(\xi_i).
	\label{eq:ySiguiente}
\end{equation}
El método de Euler construye $w_i\approx y(t_i)$ para cada $i=1,2,\dots,N$, al eliminar el término restante. Por tanto,
\begin{definition}[Método de Euler]
	\begin{align}
		w_0 &= \alpha, \nonumber\\ 
		w_{i+1} &= w_i + hf(t_i, w_i) \nonumber\\
		&\forall \, i=0,1,2,\dots ,N-1. 
		\label{eq:ecuacionEnDiferenicas}
	\end{align}
\end{definition}

A la ecuación \ref{eq:ecuacionEnDiferenicas} se le llama \textbf{ecuación en diferencias} asociada al método de Euler. En el algoritmo 
\ref{algo:Euler} se pone en ejecución el método de Euler.

\begin{algorithm}[h]
  	\SetKwInOut{Input}{Entrada}
  	\SetKwInOut{Output}{Salida}
  	\SetKwFunction{Salida}{Salida}
	\SetKwFunction{euler}{Euler}	
	\Input{$a,b,N,\alpha$}
	\Output{$w_N$.}
	\euler ($a,b,N,\alpha$) \Begin {
		$h=\dfrac{b-a}{N}$\\
		$w_0=\alpha$\\  		 
  		\For{$i=0$ \KwTo $N-1$}{
  		    $t=a+ih$\\
  			$w=w_0+hf(t,w_0)$\\
  			$w_0 = w$\\
  		}
		\Salida($w$)\\
  	}
\caption{Método de Euler}
\label{algo:Euler}
\end{algorithm}

Para interpretar geométricamente el método de Euler, observe que cuando $w_i$ es una aproximación cercana a $y(t_i)$, la suposición 
de que el problema está bien planteado implica que 
\[f(t_i,w_i) \approx y'(t_i) = f(t_i, y(t_i)).\]
En la figura \ref{fig:funcionEuler} aparece la gráfica de la función, donde se resalta $y(t_i)$. En la figura \ref{fig:pasoMetodoEuler} 
se muestra un paso del método de Euler y en la figura \ref{fig:seriePasosMetodoEuler} se ve una serie de pasos. 

\begin{figure}[H]
  \centering
  \includegraphics[scale=0.2]{../img/euler1.png}
  \caption{Gráfica de la función}
  \label{fig:funcionEuler}
\end{figure}

\begin{figure}[H]
  \centering
  \includegraphics[scale=0.2]{../img/euler2.png}
  \caption{Paso del Método de Euler}
  \label{fig:pasoMetodoEuler}
\end{figure}

\begin{figure}[H]
  \centering
  \includegraphics[scale=0.2]{../img/euler3.png}
  \caption{Serie de pasos del Método Euler}
  \label{fig:seriePasosMetodoEuler}
\end{figure}


\begin{exerciseT}{\rm Utilice el método de Euler con $N=10$ para encontrar una aproximación a la solución del problema
de valores iniciales en $t=2$. 
\begin{align*}
	y' &= y - t^2 +1\\
	0&\leq t\leq 2\\
	y(0) &= 0.5
\end{align*}

\subsubsection*{Solución.}
Dado que $N=10$, entonces
\begin{align*}
	h &= \dfrac{b-a}{N} = \dfrac{2-0}{10} = 0.2 \\
	t_i &= 0.2i \\
	w_0 &= 0.5\\
	w_{i+1} &= w_i + h(w_i - t_i^2 + 1)\\ 
		&= w_i + 0.2[w_i - 0.04i^2 + 1]\\ 
		&= 1.2w_i - 0.008i^2 + 0.2
\end{align*}
para $i=0,1,\dots,9$. Por lo tanto
\[\begin{array}{ll}
	\mbox{$i=0: $} & w_1 = 1.2(0.5) - 0.008(0)^2 + 0.2 = 0.8 \\
	\mbox{$i=0: $} & w_2 = 1.2(0.8) - 0.008(1)^2 + 0.2 = 1.152
\end{array}\]


y así sucesivamente. En la tabla \ref{table:resultadosEjemplo2Euler} se muestra la comparación entre los valores aproximados en 
$t_i$ y los valores reales.

	\begin{table}[H]
    	\centering
      	\begin{tabular}{ccc}
      		\toprule
      		$i$ & $t_i$ & $w_{i+1}$ \\
      		\midrule
			0 & 0.0 & 0.8000000 \\
			1 & 0.2 & 1.1520000 \\
			2 & 0.4 & 1.5504000 \\
			3 & 0.6 & 1.9884800 \\
			4 & 0.8 & 2.4581760 \\
			5 & 1.0 & 2.9498112 \\
			6 & 1.2 & 3.4517734 \\
			7 & 1.4 & 3.9501281 \\
			8 & 1.6 & 4.4281538 \\
			9 & 1.8 & 4.8657845 \\
			\bottomrule
      	\end{tabular}
      	\caption{Resultados de las aproximaciones del Ejemplo \ref{ex:euler2}}
      	\label{table:resultadosEjemplo2Euler}
  	\end{table} 
  	
Finalmente, la aproximación obtenida por el método es \[y(2) \approx w_{10} = 4.8657845.\]
\label{ex:euler2}
}\end{exerciseT}

\section{Método de Euler mejorado}
El método de Euler mejorado corresponde a $a_1=a_2=\frac{1}{2}$ y $\alpha_2 = \delta_2 = h$ y presenta la siguiente forma de 
ecuación en diferencias que corresponde a un método $O(h^2)$.

\begin{definition}[Método de Euler mejorado]
	\begin{align*}
		w_0 &= \alpha,\\
		w_{i+1} &= w_i + \dfrac{h}{2}\left[f(t_i,w_i) + f(t_{i+1}, w_i + hf(t_i,w_i))\right], && \mbox{para } i=0,1,\dots,N-1.
	\end{align*}
\end{definition}

\begin{algorithm}[h]
  	\SetKwInOut{Input}{Entrada}
  	\SetKwInOut{Output}{Salida}
  	\SetKwFunction{Salida}{Salida}
	\SetKwFunction{eulermejorado}{EulerMejorado}	
	\Input{$a,b,N,\alpha$}
	\Output{$w_N$.}
	\eulermejorado ($a,b,N,\alpha$) \Begin {
		$h=\dfrac{b-a}{N}$\\
		$w_0=\alpha$\\  		 
  		\For{$i=0$ \KwTo $N-1$}{
  		    $t_i=a+ih$\\
  			$w=w_0+\dfrac{h}{2}\left[f(t_i,w_0) + f(t_{i+1}, w_0 + hf(t_i,w_0))\right]$\\
  			$w_0=w$\\
  		}
		\Salida($w$)\\
  	}
\caption{Método de Euler mejorado}
\label{algo:EulerMejorado}
\end{algorithm}


\begin{exerciseT}{\rm
	Utilice el método de Euler mejorado con $N=10$ para aproximar la solución del Problema de Valores Iniciales.
	\begin{align*}
		y'=y-t^2+1, && 0\leq t\leq 2, && y(0)=0.5
	\end{align*}
	\begin{align*}
		\intertext{Planteamiento}
		h &= \dfrac{b-a}{N} = \dfrac{2-0}{10} = \dfrac{1}{5}\\
		t_i &= a+ih = 0+ i\dfrac{1}{5} = \dfrac{i}{5}\\
		w_0 &= 0.5\\
		f(t_i,w_i) &= w_i-t_i^2+1
	\end{align*}
	\begin{align*}
		w_{i+1} &= w_i + \dfrac{h}{2}\left[f(t_i,w_i) + f(t_{i+1}, w_i + hf(t_i,w_i))\right], && \forall i=0,1,\dots,N-1\\
		f(t_{i+1}, w_i + hf(t_i,w_i)) &= f\left(t_{i+1}, w_i + \dfrac{1}{5}\left[w_i - t_i^2 + 1 \right]\right)\\
			&= f\left(\dfrac{i+1}{5}, w_i + \dfrac{1}{5}w_i - \dfrac{(i/5)^2}{5} + \dfrac{1}{5} \right)\\
			&= f\left(\dfrac{i}{5}+\dfrac{1}{5}, \dfrac{6}{5}w_i - \dfrac{i^2}{125} + \dfrac{1}{5} \right)\\
			&= \dfrac{6}{5}w_i - \dfrac{i^2}{125} + \dfrac{1}{5} - \left(\dfrac{i}{5}+\dfrac{1}{5}\right)^2+1\\
			&= \dfrac{6}{5}w_i - \dfrac{6}{125}i^2 - \dfrac{2}{25}i + \dfrac{29}{25}\\
		w_{i+1} &= w_i + \dfrac{1/5}{2}\left[w_i - \left(\dfrac{i}{5}\right)^2 + 1 + \dfrac{6}{5}w_i - \dfrac{6}{125}i^2 - \dfrac{2}{25}i 
			+ \dfrac{29}{25} \right]\\
		w_{i+1} &= \dfrac{61}{50}w_i - \dfrac{11}{1250}i^2 - \dfrac{1}{125}i + \dfrac{27}{125}\\
		w_{i+1} &= 1.22w_i - 0.0088i^2 - 0.008i + 0.216 && \forall i=0,1,\dots,9
	\end{align*}
	Obteniendo las aproximaciones para cada $i=0,1,\dots ,9$.
	\[\begin{array}{ll}
		i=0: & w_1 = \frac{61}{50}(0.5) - \frac{11}{1250}(0^2) - \frac{1}{125}(0) + \frac{27}{125} = 0.826\\
		i=1: & w_2 = \frac{61}{50}(0.826) - \frac{11}{1250}(1^2) - \frac{1}{125}(1) + \frac{27}{125} = 1.20692
	\end{array}\]
	\begin{table}[H]
		\centering
		\begin{tabular}{ccc}
			\toprule
			$i$ & $t_i$ & $w_{i+1}$\\
			\midrule
			0 & 0.0 & 0.826\\
			1 & 0.2 & 1.20692\\
			2 & 0.4 & 1.63724\\
			3 & 0.6 & 2.11024\\
			4 & 0.8 & 2.61769\\
			5 & 1.0 & 3.14958\\
			6 & 1.2 & 3.69369\\
			7 & 1.4 & 4.2351\\
			8 & 1.6 & 4.75562\\
			9 & 1.8 & 5.23305\\
			\bottomrule
		\end{tabular}
		\caption{Aproximaciones del método de Euler mejorado.}
	\end{table}
	
La aproximación obtenida por el método es \[y(2) \approx w_{10} = 5.23305.\]	
}\end{exerciseT}

\section{Método de Ruge-Kutta}
Los métodos de Runge-Kutta tienen error de truncamiento local de orden alto, como los métodos de Taylor, pero permiten prescindir 
del cálculo y evaluación de las derivadas de $f(t,y)$. 

El primer paso al deducir el método de Runge-Kutta, es determinar los valores de $a_1$, $\alpha_1$ y $\beta_a$, con la propiedad de 
que $a_1f(t+\alpha_1, y+\beta_1)$ aproxima
\[ T^{(2)}(t,y) = f(t,y) + \dfrac{h}{2}f'(t,y), \]
con un error no mayor que $O(h^2)$, el cual es igual que el error de truncamiento local del método de Taylor de orden dos.

El método de la ecuación en diferencias que resulta al sustituir $T^{(2)}(t,y)$ en el método de Taylor de orden dos por 
$f\left(t+\left(\dfrac{h}{2}\right), y + \left(\dfrac{h}{2}\right)f(t,y) \right)$ es un método específico de Runge-Kutta, conocido 
como \textit{método del punto medio}.

\begin{definition}[Método de Runge Kutta: punto medio]
	\begin{align*}
		w_0 &= \alpha,\\
		w_{i+1} &= w_i + hf\left(t_i + \dfrac{h}{2}, w_i + \dfrac{h}{2}f(t_i,w_i) \right) && \mbox{para } i=0,1,\dots,N-1.
	\end{align*}
\end{definition}

El algoritmo \ref{algo:RungeKuttaPuntoMedio} muestra el procedimiento para implementar el método de Runge-Kutta del punto medio.

\begin{algorithm}[h]
  	\SetKwInOut{Input}{Entrada}
  	\SetKwInOut{Output}{Salida}
  	\SetKwFunction{Salida}{Salida}
	\SetKwFunction{rungekutta}{Runge-Kutta}	
	\Input{$a,b,N,\alpha$}
	\Output{$w_N$.}
	\rungekutta ($a,b,N,\alpha$) \Begin {
		$h=\dfrac{b-a}{N}$\\
		$w_0=\alpha$\\  		 
  		\For{$i=0$ \KwTo $N-1$}{
  		    $t_i=a+ih$\\
  			$w=w_0+hf\left(t_i + \dfrac{h}{2}, w_0 + \dfrac{h}{2}f(t_i,w_0) \right)$\\
  			$w_0=w$\\
  		}
  		\Salida($w$)\\
  	}
\caption{Método de Runge Kutta: Punto medio}
\label{algo:RungeKuttaPuntoMedio}
\end{algorithm}

\begin{exerciseT}{\rm
	Utilice el método de Runge Kutta del punto medio con $N=10$ para aproximar la solución del Problema de Valores Iniciales.
	\begin{align*}
		y'=y-t^2+1, && 0\leq t\leq 2, && y(0)=0.5
	\end{align*}
	\begin{align*}
		h &= \dfrac{b-a}{N} = \dfrac{2-0}{10} = \dfrac{1}{5}\\
		t_i &= a+ih = 0+ i\left(\dfrac{1}{5}\right) = \dfrac{i}{5}\\
		w_0 &= \alpha = 0.5\\
		f(t_i,w_i) &= w_i-t_i^2+1 = w_i - \dfrac{i^2}{25} + 1
	\end{align*}
	\begin{align*}
		w_{i+1} &= w_i + hf\left(t_i + \dfrac{h}{2}, w_i + \dfrac{h}{2}f(t_i,w_i) \right) && \mbox{para } i=0,1,\dots,N-1.\\
		f\left(t_i+\frac{h}{2}, w_i+\frac{h}{2}f(t_i,w_i)\right) &= f\left(\dfrac{i}{5}+\frac{1}{10}, w_i + \frac{1}{10}f(t_i,w_i)\right)\\
			&= f\left(\dfrac{i}{5}+\frac{1}{10}, w_i + \dfrac{1}{10}\left[w_i-t_i^2+1\right] \right)\\
			&= f\left(\dfrac{i}{5}+\frac{1}{10}, \dfrac{11}{10}w_i - \dfrac{i^2}{250} + \dfrac{1}{10} \right)\\
			&= \dfrac{11}{10}w_i - \dfrac{i^2}{250} + \dfrac{1}{10} - \left(\dfrac{i}{5}+\frac{1}{10}\right)^2 + 1\\
			&= \dfrac{11}{10}w_i - \dfrac{11}{250}i^2 - \dfrac{i}{25} + \dfrac{109}{100}\\
		w_{i+1} &= w_i + \dfrac{1}{5}\left[\dfrac{11}{10}w_i - \dfrac{11}{250}i^2 - \dfrac{1}{25}i + \dfrac{109}{100} \right]\\
		w_{i+1} &= \dfrac{61}{50}w_i - \dfrac{11}{1250}i^2 - \dfrac{1}{125}i + \dfrac{109}{500} && \forall i=0,1,\dots,9\\
		w_{i+1} &=1.22w_i - 0.008i^2 - 0.0088i + 0.218 && \forall i=0,1,\dots,9
	\end{align*}
	\begin{align*}
		\intertext{Obteniendo las aproximaciones para cada $i=0,1,\dots ,9$.}
		\mbox{$i=0:$} && w_1 &= \dfrac{61}{50}(0.5) - \dfrac{11}{1250}(0^2) - \dfrac{1}{125}(0) + \dfrac{109}{500} = 0.828\\
		\mbox{$i=1$:} && w_2 &= \dfrac{61}{50}(0.828) - \dfrac{11}{1250}(1^2) - \dfrac{1}{125}(1) + \dfrac{109}{500} = 1.21136\\
	\end{align*}
	\begin{table}[H]
		\centering
		\begin{tabular}{c|c|c}
		    \toprule
			$i$ & $t_i$ & $w_{i+1}$\\
			\midrule
			0 & 0.0 & 0.828 \\
			1 & 0.2 & 1.21136 \\
			2 & 0.4 & 1.64466 \\
			3 & 0.6 & 2.12128 \\
			4 & 0.8 & 2.63317 \\
			5 & 1.0 & 3.17046 \\
			6 & 1.2 & 3.72117 \\
			7 & 1.4 & 4.27062 \\
			8 & 1.6 & 4.80096 \\
			9 & 1.8 & 5.29037 \\
			\bottomrule
		\end{tabular}
		\caption{Aproximaciones del método de Runge-Kutta: Punto medio}
	\end{table}
    La aproximación obtenida por el método es \[y(2) \approx w_{10} = 5.29037.\]	
}\end{exerciseT}



% \section{Método de pasos múltiples}
% \section{Sistemas de ecuaciones diferenciales ordinarias}
\section{Aplicaciones}

\subsection{Crecimiento Poblacional}

Uno de los primeros intentos para modelar el \textbf{crecimiento de la población} humana por medio de las matemáticas fue realizado en 
1798 por el economista inglés Thomas Malthus. Básicamente la idea detrás del modelo de Malthus es la suposición de que la rapidez con 
la que crece una población con respecto al tiempo es directamente proporcional a la población en ese momento. En otras palabras, entre 
más personas estén presentes al tiempo $t$, habrá más en el futuro. En términos matemáticos, la ecuación \ref{eq:proporcionPoblacional} 
muestra esta relación.

\begin{equation}
	\dfrac{dP}{dt} \propto P
	\label{eq:proporcionPoblacional}
\end{equation}

donde $P(t)$ es la población para algún tiempo $t$ y $\dfrac{dP}{dt}$ es la razón de cambio de $P$ con respecto al tiempo $t$.

Para convertir esta relación de proporcionalidad en una ecuación se agrega una constante de proporcionalidad $k$. De esta forma, la ecuación 
resultante \ref{eq:modeloPoblacional} que es el modelo de crecimiento poblacional será:

\begin{equation}
	\dfrac{dP}{dt} = kP(t)
	\label{eq:modeloPoblacional}
\end{equation}

La constante de proporcionalidad indica el comportamiento de la población, si $k>0$ será un crecimiento en la población, si $k<0$ será un 
decrecimiento y si $k=0$ entonces no habrá cambios en la población. En este modelo, dicha constante esta dada por la expresión 
\ref{eq:constanteMPoblacional}.

\begin{equation}
	k = \frac{\ln \left(\dfrac{P(t)}{P_0} \right)}{t}
	\label{eq:constanteMPoblacional}
\end{equation}

Este modelo solo considera la reproducción de la población existente y no toma en cuenta  muchos otros factores que pueden influir en el 
crecimiento o decrecimiento, como por ejemplo inmigración y emigración. 

\begin{exerciseT}{\rm La población de un pueblo crece con una razón proporcional a la población en el tiempo $t$. La población inicial 
de 500 aumenta 15\% en 10 años. ¿Cuál será la población a los 30 años?

\subsubsection*{Solución.}
	El modelo aplicado en este problema queda:
	$$\dfrac{dP}{dt} = kP, \, 0\leq t\leq 30$$
	La condición inicial $P(0)=500$ y la condición $P(10)=575$.\\
	Calculando la constante de proporcionalidad del modelo de acuerdo a la ecuación \ref{eq:constanteMPoblacional}:	
	$$k=\frac{ln \left(\dfrac{575}{500} \right)}{10} = 0.0139761.$$
	Planteando el problema para el método de Euler:
	$$N=5,$$
	$$h=\frac{b-a}{N} = \frac{30-0}{5} = 6,$$
	$$t_0 = a = 0,$$
	$$w_0 = \alpha = 500,$$
	$$t_i = a+ih = 0 + i(6) = 6i,$$
	$$w_{i+1} = w_i + hf(t_i,w_i) = w_i + 6[kw_i] = 1.0838w_i.$$
	Cabe mencionar que es importante que le ecuación diferencial se encuentre en la forma normal para poder determinar la función $f(t,y)$ 
	(en este caso $f(t,P)$).
	
	Ahora solo sigue realizar las $N=5$ iteraciones del método de Euler, dichos resultados se encuentran en la tabla \ref{table:resEjPoblacional}
	
	\begin{table}[H]
    	\begin{center}
      	\begin{tabular}{c|c}
      		\toprule
      		$t_i$ & $w_i$\\
      		\midrule
			0 & 541.9238\\
			6 & 587.3725\\
			12 & 636.6273\\
			18 & 690.01305\\
			24 & 747.8752\\
			\bottomrule
      	\end{tabular}
      	\caption{Resultados de las aproximaciones del Ejemplo \ref{ex:ejPoblacional}}
      	\label{table:resEjPoblacional}
    	\end{center}
  	\end{table} 

	Luego de 5 iteraciones la aproximación obtenida por el método es $P(30) = 747.8752$.\\
	Al repetir el método de Euler con $N=100$ la aproximación es $P(30) = 759.7690$, y con $N=1000$ la aproximación es $P(30) = 760.3685$. 
	Por otro lado, el resultado exacto obtenido a través de la solución analítica de la ecuación diferencial es $P(30) = 760$.
	\label{ex:ejPoblacional}
}\end{exerciseT}

\subsection{Decaimiento Radiactivo - Vida Media}

El núcleo de un átomo está formado por combinaciones de protones y neutrones. Muchas de esas combinaciones son inestables, esto es, los 
átomos se desintegran o se convierten en átomos de otras sustancias. Se dice que estos núcleos son radiactivos. Por ejemplo, con el tiempo, 
el radio Ra 226, intensamente radiactivo, se transforma en el radiactivo gas radón, Rn-222. Para modelar el fenómeno del \textbf{decaimiento 
radiactivo}, se supone que la razón $dA/dt$ con la que los núcleos de una sustancia se desintegran es proporcional a la cantidad (más precisamente, 
el número de núcleos), $A(t)$ de la sustancia que queda en el tiempo $t$:

\begin{equation}
	{dA\over dt} = kA(t)
	\label{eq:vidaMedia}
\end{equation}

El decaimiento de sustancias radiactivas se comporta de acuerdo al modelo poblacional (ecuación \ref{eq:modeloPoblacional}). Para este caso 
$A(t)$ indica la cantidad de sustancia radiactiva existente en el tiempo $t$, y la constante de proporcionalidad es estrictamente $k<0$. 
La ecuación \ref{eq:vidaMedia} muestra el modelo.

La vida media se define como el tiempo $t_m$ necesario para que una sustancia radiactiva se reduzca a la mitad. 

La constante de proporcionalidad también se obtiene con la ecuación \ref{eq:constanteMPoblacional} debido a que el modelo es muy similar al 
Poblacional.

\begin{exerciseT}{\rm Se encuentra que un hueso fosilizado y se estima que su edad es de 55,800 años. Determine la cantidad de C-14 que 
tiene el fósil al momento del descubrimiento si se sabe que (inicialmente) al momento de su muerte había 100 gramos de C-14.
	
	\subsubsection*{Solución}
	El modelo aplicado en este problema queda:
	\begin{align*}
	    \dfrac{dA}{dt} = kA && 0\leq t\leq 55,800
	\end{align*}
	La condición inicial $A(0)=100$ y la vida media del C-14 determina la condición $A(5,600)=\displaystyle{\frac{A_0}{2}}$.\\
	Calculando la constante de proporcionalidad del modelo de acuerdo a la ecuación \ref{eq:constanteMPoblacional}:	
	$$k=\frac{ln \left(\displaystyle{\frac{50}{100}} \right)}{5600} = -0.000123776.$$
	Planteando el problema para el método de Euler:
	\begin{align*}
	    N &= 10\\
	    h &= \dfrac{b-a}{N} = \dfrac{55800-0}{10} = 5580 \\
	    t_i &= a + ih = 5580i\\
        w_0 &= \alpha = 100\\
        w_{i+1} &= w_i + hf(t_i,w_i) = w_i + 5580[kw_i] = w_i-0.69067165w_i = 0.30932835w_i.
	\end{align*}
	
	Ahora solo sigue realizar las $N=10$ iteraciones del método de Euler, dichos resultados se encuentran en la tabla \ref{table:resEjVidaMedia}
	
	\begin{table}[H]
    	\centering
      	\begin{tabular}{c|c}
      		\toprule
      		$t_i$ & $w_{i+1}$\\
      		\midrule
			0 		& 30.93\\
			5,580   & 9.56\\
			11,160 	& 2.95\\
			16,740  & 0.9155\\
			22,320 	& 0.2832\\
			27,900  & 0.0876\\
			33,480  & 0.0270\\
			39,060  & 0.0083\\
			44,640	& 0.0025\\
			50,220  & 0.0008\\
			\bottomrule
      	\end{tabular}
      	\caption{Resultados de las aproximaciones del Ejemplo \ref{ex:ejVidaMedia}}
      	\label{table:resEjVidaMedia}
  	\end{table} 
	
	Luego de $N=10$ iteraciones la aproximación del método es $A(55,800) = 0.0008g$.
	\label{ex:ejVidaMedia}
}\end{exerciseT}

\subsection{Ley de Enfriamiento y Calentamiento de Newton}

De acuerdo con la ley empírica de Newton de enfriamiento/calentamiento, la rapidez con la que cambia la temperatura de un cuerpo es proporcional
a la diferencia entre la temperatura del cuerpo y la del medio que lo rodea, que se llama temperatura ambiente. Si $T(t)$ representa la 
temperatura del cuerpo al tiempo $t$, $T_m$ es la temperatura constante del medio que lo rodea y $dT/dt$ es la rapidez con que cambia la 
temperatura del cuerpo, entonces la ley de Newton de enfriamiento/calentamiento traducida en una expresión matemática es 

\begin{equation}
	{dT\over dt} = k(T-T_m)
	\label{eq:leyEnfCalNewton}
\end{equation}

donde $k$ es una constante de proporcionalidad. En ambos casos, enfriamiento o calentamiento, si $T_m$ es una constante, se establece que $k<0$.

\begin{equation}
	k = \frac{\ln \left( \dfrac{T(t)-T_m}{T(0)-T_m} \right)}{t}
	\label{eq:constanteLECNewton}
\end{equation}

\begin{exerciseT}{\rm Un termómetro se cambia de una habitación donde la temperatura es de $70\ensuremath{^\circ}F$ al exterior, donde 
la temperatura del aire es de $10\ensuremath{^\circ}F$. Después de medio minuto el termómetro indica $50\ensuremath{^\circ}F$. ¿Cuál es 
la lectura del termómetro en $t=1$ y $t=10$ minutos?

	\subsubsection*{Solución}
	El modelo aplicado en este problema queda:
	$$\dfrac{dT}{dt} = k(T-T_m), \, 0\leq t\leq 10$$
	La condición inicial $T(0)=70$, $T(1/2)=50$ y $T_m=10$.\\
	Calculando la constante de proporcionalidad del modelo de acuerdo a la ecuación \ref{eq:constanteLECNewton}:	
	$$k=\frac{\ln \left(\dfrac{50-10}{70-10} \right)}{\frac{1}{2}} = -0.81093.$$
	Planteando el problema para el método de Euler:
	$$N=10,$$
	$$h=\frac{b-a}{N} = \frac{10-0}{10} = 1,$$
	$$t_0 = a = 0,$$
	$$w_0 = \alpha = 70,$$
	$$t_i = a+ih = 0 + i(1) = i,$$
	$$w_{i+1} = w_i + hf(t_i,w_i) = w_i + 1[k(w_i-10)] = 0.189069w_i + 8.1093.$$
	
	Ahora solo sigue realizar las $N=10$ iteraciones del método de Euler, dichos resultados se encuentran en la tabla \ref{table:resEjTemperatura}
	
	\begin{table}[H]
	    \centering
      	\begin{tabular}{c|c}
      		\toprule
      		$t_i$ & $w_{i+1}$\\
      		\midrule
      		0 & 21.344200000000001\\
			1 & 12.144847894000000\\
			2 & 10.405526391318579\\
			3 & 10.076672874806604\\
			4 & 10.014496540439685\\
			5 & 10.002740860900932\\
			6 & 10.000518214570539\\
			7 & 10.000097978828851\\
			8 & 10.000018524857170\\
			9 & 10.000003502494746\\
			\bottomrule
      	\end{tabular}
      	\caption{Resultados de las aproximaciones del Ejemplo \ref{ex:ejTemperatura}}
      	\label{table:resEjTemperatura}
  	\end{table} 
	
	Luego de $N=10$ iteraciones las aproximaciones del método son $T(1) = 21.344200000000001$ y $T(10) = 10.000003502494746$.
	\label{ex:ejTemperatura}
}\end{exerciseT}

\subsection{Propagación de Enfermedades}

Una enfermedad contagiosa, por ejemplo un virus de gripe, se propaga a través de una comunidad por personas que han estado en contacto 
con otras personas enfermas. Sea que $x(t)$ denote el número de personas que han contraído la enfermedad y sea que $y(t)$ denote el número 
de personas que aún no han sido expuestas al contagio. Es lógico suponer que la razón $dx/dt$ con la que se propaga la enfermedad es 
proporcional al número de encuentros, o interacciones, entre estos dos grupos de personas. Si suponemos que el número de interacciones 
es conjuntamente proporcional a $x(t)$ y $y(t)$, esto es, proporcional al producto $xy$, entonces

\begin{equation}
	\dfrac{dx}{dt} = kxy,
	\label{eq:propEnfermedades}
\end{equation}

donde $k$ es la constante usual de proporcionalidad. 

Suponga que una pequeña comunidad  tiene una población fija de $n$ personas. Si se introduce una persona infectada dentro de esa comunidad, 
entonces se podría argumentar que $x(t)$ y $y(t)$ están relacionadas por $x+y=n+1$. Utilizando esta última ecuación para eliminar $y$ en 
la ecuación \ref{eq:propEnfermedades} se obtiene el modelo

\begin{equation}
	\dfrac{dx}{dt} = kx(n+1-x)
	\label{eq:propEnfermedades2}
\end{equation}

Una condición inicial obvia que acompaña a la ecuación \ref{eq:propEnfermedades2} es $x(0)=1$.

\subsection*{Ejercicios}

\subsubsection{Modelo Poblacional}

\begin{enumerate}
	\item Se sabe que la población de una comunidad crece con una razón proporcional al número de personas presentes en el tiempo $t$. 
	    Si la población inicial $P_0 = 1000$ personas se duplicó en 5 años, ¿en cuánto tiempo se triplicará y cuadruplicará?
	\item La población de bacterias en un cultivo crece a una razón proporcional a la cantidad de bacterias presentes al tiempo $t$. 
	    Suponga que inicialmente había 1 bacteria y después de tres horas se observa que hay 400 bacterias presentes. ¿Cuál es la 
	    cantidad de bacterias luego de 10 horas?. 
\end{enumerate}


\subsubsection{Ley de Enfriamiento/Calentamiento de Newton}

\begin{enumerate}
	\item Una pequeña barra de metal, cuya temperatura inicial era de 20\ensuremath{^\circ}C, se deja caer en un gran tanque de agua 
	    hirviendo. ¿Cuánto tiempo tardará la barra en alcanzar los 90\ensuremath{^\circ}C si se sabe que su temperatura aumentó 
	    2\ensuremath{^\circ}C en 1 segundo?.
	\item Un termómetro que indica 70\ensuremath{^\circ}F se coloca en un horno precalentado a una temperatura constante de 
	    180\ensuremath{^\circ}C. A través de una ventana de vidrio en la puerta del horno, un observador registra que el termómetro lee
	    110\ensuremath{^\circ}F, ¿cuál será la temperatura que registrará el termómetro a los 5 y 10 minutos?
\end{enumerate}

\subsubsection{Decaimiento Radiactivo}

\begin{enumerate}
	\item Inicialmente había 100 miligramos de una sustancia radiactiva. Después de 6 horas la masa disminuyó 3\%. Si la razón de 
	    decaimiento, en cualquier momento, es proporcional a la cantidad de la sustancia presente al tiempo $t$, determine la cantidad 
	    que queda después de 24 horas. 
	\item El isótopo radiactivo del plomo Pb-209, decae con una razón proporcional a la cantidad presente al tiempo $t$ y tiene una 
	    vida media de 3.3 horas. Si al principio había 1 gramo de plomo, ¿cuánto sustancia queda luego de 10 horas?
\end{enumerate}
