\chapter{Solución de ecuaciones}
Uno de los problemas básicos de la aproximación numérica es la búsqueda de raíces. Consiste en obtener una raíz, 
o una solución, de una ecuación de la forma $f(x)=0$ para una función dada $f$. Una raíz de esta ecuación es 
llamada también un cero de la función $f$. 

\section{Método de Newton Raphson}
El método de Newton se basa en los polinomios de Taylor, este tipo particular de deducción produce no sólo el método, 
sino también una cota para el error de aproximación.\\
Supongamos que $f\in C^{2}[a, b]$. Sea $p_{0} \in [a, b]$ una aproximación de $p$ tal que $f'(p_{0}) \not= 0$ y 
$|p - p_{0}|$ es "pequeño". Considere el primer polinomio de Taylor para $f(x)$ expandido alrededor de $p_{0}$ y 
evaluado en $x=p$.

$$f(p) = f(p_{0}) + (p-p_{0})f'(p_{0}) + \frac{(p-p_{0})^{2}}{2}f''(\xi(p))$$

donde $\xi(p)$ está entre $p$ y $p_{0}$. Dado que $f(p)=0$ esta ecuación da como resultado

$$0 = f(p_{0}) + (p-p_{0})f'(p_{0}) + \frac{(p-p_{0})^{2}}{2}f''(\xi(p))$$

El método de Newton se obtiene suponiendo que, como $|p-p_{0}|$ es tan pequeño, el término que contiene $(p-p_{0})^{2}$ 
es mucho menor y que

$$0 \approx f(p_{0}) + (p-p_{0}))f'(p_{0})$$

Despejando $p$ de esta ecuación obtenemos

$$p\approx p_{0} - \frac{f(p_{0})}{f'(p_{0})} \equiv p_{1}$$

Esto nos prepara para introducir el método de Newton, el cual comienza con una aproximación inicial $p_{0}$ y genera 
la sucesión $\{ p_{n}\}_{n=0}^{\infty}$ definida por

\begin{theorem}
Dada una función $f(x)$ definida $\forall x\in C$ donde $C\subseteq D(f)$. Además $f(x)$ es derivable y $f'(x)=0$ 
$\forall x\in C$. Sea $p_0\in\mathbb{R}$ y $p_0\in C$ entonces:  
	\[p_{n} = p_{n-1} - \frac{f(p_{n-1})}{f'(p_{n-1})},\, n\geq 1\]
	\label{eq:Newton}
\end{theorem}

La figura \ref{fig:metodoNewton} muestra gráficamente como se obtienen las aproximaciones usando tangentes sucesivas. 
Comenzando con la aproximación inicial $p_{0}$, la aproximación $p_{1}$ es la intersección con el eje x de la recta 
tangente a la gráfica de $f$ en $(p_{0}, f(p_{0}))$. La aproximación $p_{2}$ es la intersección con el eje x de 
la recta tangente a la gráfica de $f$ en $(p_{1}, f(p_{1}))$ y así sucesivamente. El algoritmo \ref{algo:Newton} 
sigue ese procedimiento.  

\begin{figure}[H]
	\centering
    \includegraphics[scale=0.17]{../img/newton1.jpg}
    \caption{Método de Newton-Raphson}
	\label{fig:metodoNewton}
\end{figure}   

\begin{algorithm}[H]
	\SetKwInOut{Input}{input}
  	\SetKwInOut{Output}{output}
  	\SetKwFunction{Salida}{Salida}
	\SetKwFunction{Newton}{Newton}

	\Newton($p_{0},\, TOL,\, N_{0}$) \Begin {	
		$i\leftarrow 1$\\
  		 		 
  		\While{$i\leq (N_{0})$}{ 
		      $p\leftarrow p_{0} - f(p_{0})/f'(p_{0})$\\
		       \lnl{newton.1} \If{$|p-p_{0}|<TOL$}{
			\Salida(p)\\
		      }
		      $i\leftarrow i+1$\\
		      $p_{0}\leftarrow p$\\
  		}
  		\Salida(``El método fracasó después de $N_{0}$ iteraciones'')
  	}
\caption{Método de Newton-Raphson}
\label{algo:Newton}
\end{algorithm}

\begin{exerciseT}
Considere la función $f(x)= \cos (x) - x$. Aproxime una raíz de $f$ utilizando el método de Newton.

\subsection*{Solución.} 
Para resolver el problema es necesario encontrar la derivada de $f(x)$, plantear la expresión del teorema \ref{eq:Newton}
específica de este problema, elegir un valor inicial $p_0$ y realizar las aproximaciones hasta cumplir con 
la tolerancia $10^{-2}$.
\subsubsection*{Planteamiento}
Condiciones:
\begin{itemize}
	\item $f'(x) = {-\sin p_{n-1} -1 }$
	\item $f$ tiene la menos una raíz	
	\item $p_{n} = p_{n-1} - \dfrac{f(p_{n-1})}{f'(p_{n-1})} = p_{n-1} - \dfrac{\cos p_{n-1} - p_{n-1}}{-\sin p_{n-1} -1} \forall n\geq 1$
	\item Se define $p_{0}=\pi/4$
\end{itemize}

\subsubsection*{Desarrollo}
Evaluando $p_0$ en la expresión para $p_n$ se obtienen progresivamente las aproximaciones posteriores mostradas en 
la tabla \ref{table:ejemplo1Newton}.
\begin{table}[H]
	\centering
    \begin{tabular}{lll}
    	\toprule
    	i & Newton\\\midrule
		1 & 0.7395361335\\
		2 & 0.7390851781\\
		3 & 0.7390851332\\
		\bottomrule
	\end{tabular}
    \caption{Iteraciones para Ejemplo \ref{ex:ejemplo1Newton}.}
	\label{table:ejemplo1Newton}
\end{table}
Se obtiene una excelente aproximación con $n=3$. Sería razonable esperar que este resultado
tenga una precisión hasta el número de cifras enumeradas debido a la coincidencia de $p_{2}$ y $p_3$.

\subsubsection*{Resultado}
Se estima el error de la aproximación mediante 	la ecuación:
\[ ER = \dfrac{|p_n-p_{n-1}|}{|p_n|}\times 100\% \]

Por lo tanto, el error de la aproximación obtenida por el método de Newton es:
\[ER = \frac{|p_{N} - p_{N-1}|}{|p_{N}|} = \frac{|0.7390851332 - 0.7390851781|}{|0.7390851332|}\times 100\% = 6.0751\times 10^{-8}\]

El resultado completo del ejercicio es:
\begin{align*}
	p_3 &= 0.7390851332 \\
	ER &= 6.075\times 10^{-8} \%
\end{align*}

\label{ex:ejemplo1Newton}
\end{exerciseT}

\subsection*{Ejercicios}
\begin{enumerate}
	\item Aplique el método de Newton para obtener soluciones con una exactitud de $10^{-4}$ para los siguientes problemas.
		\begin{enumerate}
			\item $x^{3} - 2x^{2} -5 = 0$, $x\in [1,4]$
			\item $x - \cos x = 0$, $x\in [0, \pi/2]$
			\item $x^{3} + 3x^{2} - 1 = 0$, $x\in [-3, -2]$
			\item $x - 0.8 - 0.2\sin x = 0$, $x\in [0, \pi/2]$
		\end{enumerate}
		
		\item Aplique el método de Newton para obtener soluciones con una exactitud de $10^{-5}$ para los siguientes problemas. 
			\begin{enumerate}
				\item $e^{x} + 2^{-x} + 2\cos x - 6 = 0$, para $1\leq x\leq 2$
				\item $\ln (x-1) + \cos (x-1) = 0$, para $1.3\leq x\leq 2$
				\item $2x\cos (2x) - (x-2)^2 = 0$ para $2\leq x\leq 3$ y $3\leq x\leq 4$
				\item $(x-2)^2 - \ln x = 0$ para $1\leq x\leq 2$ y $e\leq x\leq 4$
				\item $e^{x} - 3x^2 = 0$ para $0\leq x\leq 1$, $3\leq x\leq 5$
				\item $\sin x - e^{-x} = 0$ para $0\leq x\leq 1$, $3\leq x\leq 4$ y $6\leq x\leq 7$
			\end{enumerate}		
			
		\item Con el método de Newton resuelva la ecuación, con $p_0 = \pi/2$
			$$0 = \frac{1}{2} + \frac{1}{4}x^2 - x\sin x - \frac{1}{2}\cos 2x$$	 
			Itere el método de Newton hasta llegara  una exactitud de $10^{-5}$. Explique porque el resultado parece inusual 
			para el método de Newton. También resuelva la ecuación con $p_0 = 5\pi$ y $p_0 = 10\pi$. 
\end{enumerate}

\section{Método de la secante}
El método de Newton es una técnica muy poderosa, pero presenta un grave problema: la necesidad de conocer el valor de 
la derivada de $f$ en cada aproximación.
Con frecuencia es más difícil determinar $f'(x)$ y se requieren más operaciones aritméticas para calcularla que para 
obtener $f(x)$. 

Si queremos evitar el problema de evaluar la derivada en el método de Newton, introducimos una pequeña variación. Por definición:
$$f'(p_{n-1}) = \lim_{x\rightarrow p_{n-1}}\frac{f(x) - f(p_{n-1})}{x - p_{n-1}} $$
Si $p_{n-2}$ está cerca de $p_{n-1}$, entonces:
$$f'(p_{n-1}) \approx \frac{f(p_{n-2}) - f(p_{n-1})}{p_{n-2} - p_{n-1}} = \frac{f(p_{n-1}) - f(p_{n-2})}{p_{n-1} - p_{n-2}}$$
Al aplicar esta aproximación para $f'(p_{n-1})$ en la fórmula de Newton, se obtiene el teorema \ref{teo:secante}.

\begin{theorem}[Método de la secante]
	Dada una función $f(x)$ definida $\forall x\in C$ donde $C\subseteq D(f)$. Si $p_0\approx p_1 \wedge f(p_n)-f(p_{n-1})\not=0$ 
	entonces:
	\[p_{n} = p_{n-1} - \frac{f(p_{n-1}) (p_{n-1} - p_{n-2})}{f(p_{n-1}) - f(p_{n-2})}, \forall n\geq 2\]
	\label{teo:secante}
\end{theorem}

La técnica que utiliza el teorema recibe el nombre de método de la secante y se representa mediante el algoritmo 
\ref{algo:metodoSecante} (Vea la figura \ref{fig:metodoSecante}). Comenzando con las 
dos aproximaciones iniciales $p_{0}$ y $p_{1}$, la aproximación $p_2$ es la intersección con el eje $x$ de la recta que une 
$(p_0, f(p_0))$ y $(p_1, f(p_1))$. La aproximación de $p_3$ es la intersección con el eje $x$ de la recta que une $(p_1, f(p_1))$ 
y $(p_2, f(p_2))$, y así sucesivamente. Observe que solo es necesaria la evaluación de la función en cada paso del método de la 
secante después de que se ha determinado $p_2$. En contraste, cada paso del método de Newton requiere una evaluación de
la función y su derivada.

\begin{figure}[H]
	\centering
    \includegraphics[scale=0.13]{../img/metodoSecante.jpg}
    \caption{Método de la Secante}
    \label{fig:metodoSecante}
\end{figure}   

\begin{algorithm}[H]
  	\SetKwInOut{Input}{input}
  	\SetKwInOut{Output}{output}
  	\SetKwFunction{Salida}{Salida}
	\SetKwFunction{Secante}{Secante}

	\Secante($p_{0}, p_1,\, TOL,\, N_{0}$) \Begin {	
		$i\leftarrow 2$\\
		$q_0\leftarrow f(p_0)$\\
		$q_1\leftarrow f(p_1)$\\
		\While{$i\leq N_0$}{
			$p \leftarrow p_1 - q_1(p_1 - p_0)/(q_1 - q_0)$\\
			\If{$|p-p_1| < TOL$}{
				\Salida($p$)
			}
			$i\leftarrow i+1$\\
			$p_0\leftarrow p_1$\\
			$q_0\leftarrow q_1$\\
			$p_1\leftarrow p$\\
			$q_1\leftarrow f(p)$\\
		}

  		\Salida(``El método fracasó después de $N_{0}$ iteraciones'')
  	}
\caption{Método de la Secante}
\label{algo:metodoSecante}
\end{algorithm}

\begin{exerciseT}
Utilice el método de la secante para encontrar una aproximación a una raíz de la expresión $x= \cos x$. Considere que se sabe 
que la raíz está cercana a $p_0=\pi/4$. Considere una tolerancia $10^{-2}$.
\subsection*{Planteamiento}
\begin{itemize}
	\item $f(x) = x-\cos x$ es continua en $\forall x\in\mathbb{R}$
	\item $f$ tiene la menos una raíz
	\item $p_n = p_{n-1} - \dfrac{(p_{n-1}-\cos p_{n-1})(p_{n-1}-p_{n-2})}{(p_{n-1}-\cos p_{n-1})-(p_{n-2}-\cos p_{n-2})}, \forall n\geq 2$
\end{itemize}

\subsection*{Desarrollo}
Sea $p_0 = \pi/4$ y se elige $p_1 = \pi/2$. Con esto se puede desarrollar la sucesión numérica del método y presentar los 
valores la siguiente tabla.
\begin{table}[H]
	\centering
	\begin{tabular}{c|c|c}
	$i$ & $p_i$ & $f(p_i)$ \\\midrule
	0 & $\frac{\pi}{4}$ & 0.0782914\\
	1 & $\frac{\pi}{2}$ & 1.5708\\
	2 & 0.744199 & 0.00856833\\
	3 & 0.739665 & 0.000971273\\
	4 & 0.739086 & 1.09333$\times 10^{-6}$ 
	\end{tabular}
\end{table}
Dado que se cumple: 
\[\left|p_4 -p_3 \right| = 0.000579618 < 10^{-2}  \]
entonces la aproximación $p_4$ cumple la tolerancia y se acepta como aproximación a la raíz de $f(x)$.

\subsection*{Resultado}
\begin{align*}
	p_4 &= 0.739086 \\
	ER &= \dfrac{\left|p_4 -p_3 \right|}{\left|p_4\right|}\times 100\% = \dfrac{\left|0.739086-0.739665\right|}{\left|0.739086\right|}\times 100\%
		= 0.691834\%
\end{align*}
\end{exerciseT}

\subsection*{Ejercicios}
\begin{enumerate}
	\item Aplique el método de la secante para obtener soluciones con una exactitud de $10^{-4}$ para los siguientes problemas.
		\begin{enumerate}
			\item $x^{3} - 2x^{2} -5 = 0$, $x\in [1,4]$
			\item $x - \cos x = 0$, $x\in [0, \pi/2]$
			\item $x^{3} + 3x^{2} - 1 = 0$, $x\in [-3, -2]$
			\item $x - 0.8 - 0.2\sin x = 0$, $x\in [0, \pi/2]$
		\end{enumerate}
		
		\item Aplique el método de la secante para obtener soluciones con una exactitud de $10^{-5}$ para los siguientes problemas. 
			\begin{enumerate}
				\item $e^{x} + 2^{-x} + 2\cos x - 6 = 0$, para $1\leq x\leq 2$
				\item $\ln (x-1) + \cos (x-1) = 0$, para $1.3\leq x\leq 2$
				\item $2x\cos (2x) - (x-2)^2 = 0$ para $2\leq x\leq 3$ y $3\leq x\leq 4$
				\item $(x-2)^2 - \ln x = 0$ para $1\leq x\leq 2$ y $e\leq x\leq 4$
				\item $e^{x} - 3x^2 = 0$ para $0\leq x\leq 1$, $3\leq x\leq 5$
				\item $\sin x - e^{-x} = 0$ para $0\leq x\leq 1$, $3\leq x\leq 4$ y $6\leq x\leq 7$
			\end{enumerate}		
\end{enumerate}

\section{Convergencia acelerada}

\subsection{Método de Aitken}
El método $\Delta^2$ de Aitken sirve para acelerar la convergencia de una sucesión que sea linealmente convergente, prescindiendo
de su origen o aplicación. Suponga que $\left\lbrace p_n\right\rbrace_{n=0}^\infty$ es una sucesión linealmente convergente con 
un límite $p$. Para motivar la construcción de una sucesión $\left\lbrace \hat{p}_n\right\rbrace_{n=0}^\infty$ que converja más 
rápidamente a $p$ que $\left\lbrace p_n\right\rbrace_{n=0}^\infty$.

\begin{theorem}[método de Aitken]
El método de Aitken se basa en la suposición de que la sucesión $\left\lbrace \hat{p}_n\right\rbrace_{n=0}^\infty$ definida
por:
\begin{equation}
	\hat{p}_n = p_n - \dfrac{\left(p_{n+1}-p_n \right)^2}{p_{n+2}-2p_{n+1}+p_n}
	\label{eq:aitken}
\end{equation}
converge más rápidamente a $p$ que la sucesión original $\left\lbrace p_n\right\rbrace_{n=0}^\infty$.
\end{theorem}

Al aplicar una modificación del método $\Delta^2$ de Aitken a una sucesión linealmente convergente obtenida mediante la iteración 
de punto fijo, podemos acelerar la convergencia a cuadrática. Este procedimiento es el método de Steffensen y difiere un poco de 
la aplicación directa del método $\Delta^2$ de Aitken (Teorema \ref{eq:aitken}) a la sucesión de iteraciones de punto fijo
que convergen linealmente. El método de Aitken construye los términos en el orden:

\begin{align*}
	p_0,& & p_1&=g(p_0), & p_2 &= g(p_1), & \hat{p} &= \left\lbrace \Delta^2\right\rbrace(p_0),\\
	p_3&=g(p_2), & \hat{p_1} &= \left\lbrace \Delta^2\right\rbrace(p_1),\dots
\end{align*}


\subsection{Método de Steffensen}
El método de Steffensen construye los mismos primeros cuatro términos $p_0, p_1, p_2$ y $\hat{p}_0$. No obstante, en este paso
se supone que $\hat{p}_0$ es una mejor aproximación a $p$ que $p_2$ y aplica la iteración de punto fijo a $\hat{p}_0$ en lugar
de a $p_2$. Al usar esta notación, la sucesión generada es:

\begin{align*}
	p_0^{(0)}&, & p_1^{(0)}&=g\left(p_0^{(0)}\right), & p_2^{(0)}&=g\left(p_1^{(0)}\right), & 
	p_0^{(1)}&=\left\lbrace \Delta^2\right\rbrace \left(p_0^{(0)}\right), \\
	& & p_1^{(1)}&=g\left(p_0^{(1)}\right), & p_2^{(1)}&=g\left(p_1^{(1)}\right), &
	p_0^{(2)}&=\left\lbrace \Delta^2\right\rbrace \left(p_0^{(1)}\right), \\\vdots
\end{align*}

Cada tercer término de la sucesión de Steffensen es generada por la ecuación (\ref{eq:aitken}); los demás usan la iteración de 
punto fijo en el término anterior. El procedimiento se describe en el algoritmo \ref{algo:Steffensen}.

 \begin{algorithm}[H]
  	\SetKwInOut{Input}{input}
  	\SetKwInOut{Output}{output}
  	\SetKwFunction{Salida}{Salida}
	\SetKwFunction{Steffensen}{Steffensen}

	\Steffensen($p_0,TOL,N_{0}$) \Begin {	
		$i\leftarrow 1$\\
		\While{$i\leq N_0$}{
			$p_1 = g(p_0)$\\
			$p_2 = g(p_1)$\\
			$p = p_0 - \dfrac{\left(p_1-p_0\right)^2}{p_2-2p_1+p_0}$\\
			\If{$|p-p_0| < TOL$}{
				\Salida($p$)
			}
			$i\leftarrow i+1$\\
		}
  		\Salida(``El método fracasó después de $N_{0}$ iteraciones'')
  	}
\caption{Método de Steffensen}
\label{algo:Steffensen}
\end{algorithm}

\begin{exerciseT}
Utilice el método de Steffensen y la convergencia acelerada de Aikten para encontrar una aproximación a una raíz 
de la función $f(x) = x-\cos x$. Considere que se sabe que la raíz está cercana a $p_0=\pi/4$. Considere una 
tolerancia $10^{-2}$.
\subsection*{Planteamiento}
\begin{itemize}
	\item $f(x) = x-\cos x$ es continua en $\forall x\in\mathbb{R}$
	\item $f$ tiene la menos una raíz
	\item Se deben obtener todas las funciones $g$ a partir de $x-\cos x=0$
		\begin{align*}
			g_1&: \\
				x &= \cos x\\
				g_1(x) &= \cos x \\
			g_2&: \\
				x &= \arccos x\\
				g_2(x) &= \arccos x 
		\end{align*}
\end{itemize}

\subsection*{Desarrollo}
Se define $p_0^{(0)} = \pi/4$. 
\begin{align*}
	p_0^{(0)}&= \pi/4, & p_1^{(0)}&=g\left(p_0^{(0)}\right)=, & p_2^{(0)}&=g\left(p_1^{(0)}\right), & 
	p_0^{(1)}&=\left\lbrace \Delta^2\right\rbrace \left(p_0^{(0)}\right), 
\end{align*}
Para desarrollar la sucesión numérica del método, se debe elegir una función $g$ para realizar las evaluaciones
correspondientes en la expresión acelerada de Aitken (\ref{eq:aitken}). Si la sucesión converge con 
la función $g$ evaluada se habrá encontrado la aproximación a la raíz de $f$. Por otro lado, si $g$ no converge
entonces se debe elegir otra función $g$ y repetir el procedimiento. En este caso se comienza el desarrollo con 
la función $g_1$ y los resultados se muestran en la tabla: 
\begin{table}[H]
	\centering
	\begin{tabular}{c|c|c|c|c}
	$i$ & $p_0^{(i)}$ & $p_1^{(i)}$ & $p_2^{(i)}$ & $\hat{p}_0^{(i+1)}$ \\\midrule
	0 & $\frac{\pi}{4}$ & 0.707107 & 0.760245 & 0.738761 \\
	1 & 0.738761 & 0.739304 & 0.738938 & 0.739085\\
	2 & 0.739085 & & &
	\end{tabular}
\end{table}
Dado que se cumple: 
\[\left|p_0^{(2)}-p_0^{(1)} \right| = 0.000323 	< 10^{-2}  \]
entonces la aproximación $p_4$ cumple la tolerancia y se acepta como aproximación a la raíz de $f(x)$.

\subsection*{Resultado}
\begin{align*}
	p_2 &= 0.739085 \\
	ER &= 0.04391 \%
\end{align*}
\end{exerciseT}

\subsection*{Ejercicios}
\begin{enumerate}
  \item Utilice el método de Steffensen para encontrar una aproximación a la raíz de las función indicada en 
  cada inciso. Considere una exactitud de $10^{-5}$ para la realización de los cálculos.
  \begin{enumerate}
    \item $2 + \sin x - x = 0$ y utilizar $[2,3]$
    \item $x^3-2x-5$ y utilizar $[2,3]$
    \item $3x^2-e^x=0$
    \item $x - \cos x = 0$
  \end{enumerate}
  
  \item Encuentre todos los ceros (raíces) de $f(x)=x^2+10\cos x$ aplicando el método de Steffensen. 
  Considere una exactitud de $10^{-4}$.
  
  \item Un objeto que cae verticalmente en el aire está sujeto a una resistencia viscosa y también a la fuerza 
  de gravedad. Suponga que dejamos caer el objeto de masa $m$ desde una altura $s_{0}$ y que la altura del objeto 
  después de $t$ segundos es:
  $$s(t) = s_{0} - \dfrac{mg}{k}t + \dfrac{m^2g}{k^2}(1-e^{-kt/m})$$
  donde $g=32.17$ pies/$s^2$ y $k$ representa el coeficiente de resistencia del aire. Suponga que $s_{0} = 300$ 
  pies, $m=0.25$ lb y que $k=0.1$ lb-s/pies. Calcule, con una exactitud de 0.001 s, el tiempo que tarda la masa 
  en caer al suelo.
\end{enumerate}

%\section{Aplicaciones}
