\chapter{Introducción}

\section{Panorama histórico}

\section{Sistemas de numeración}

Numéricamente los errores de redondeo se relacionan de manera directa con la forma en que se guardan los números en la memoria de la computadora. 
La unidad fundamental mediante la cual se representa la información se llama \textit{término}. Ésta es una entidad que consiste en una cadena de 
\textit{dígitos binarios} o \textit{bits}. Generalmente, los números son guardados en uno más términos. Para entender cómo se realiza esto, 
estudiaremos los sistemas numéricos.

\begin{center}
  \textit{“Solo existen 10 tipos de personas, las que entienden el binario y las que no”} 
\end{center}
Un sistema de numeración es un conjunto de símbolos y reglas que permiten construir todos los números válidos. Los sistemas de numeración más comunes son el decimal (base 10), 
binario (base 2), octal (base 8) y hexadecimal (base 16).\\
Un número decimal tal como 7392 representa una cantidad igual a 7 unidades de mil, más 3 centenas, más 9 decenas, más 2 unidades. Las unidades de mil, centenas, etc., son potencias 
de 10 implícitamente indicadas por la posición de los coeficientes. Esto es:
$$7 \times 10^{3} + 3 \times 10^{2} + 9 \times 10^{1} + 2 \times 10^{0} = 7000 + 300 + 90 + 2 = 7392$$
Y para el caso del número no entero $654.32$:
$$6 \times 10^{2} + 5 \times 10^{1} + 4 \times 10^{0} + 3 \times 10^{-1}+ 2 \times 10^{-2} = 600 + 50 + 4 + 0.3 + 0.02 = 654.32$$
Se dice que el sistema de números decimales tiene base 10 debido a que usan diez dígitos, los números del 0 al 9, y que los coeficientes son multiplicados por potencias de 10. 
Por otro lado, el sistema de numeración binario sólo tiene dos elementos: 0 y 1. Por ejemplo, el equivalente decimal del número binario 11010.11 es 26.75 como se demuestra en la 
multiplicación de los coeficientes por potencias de 2:
$$1 \times 2^{4} + 1 \times 2^{3} + 0 \times 2^{2} + 1 \times 2^{1} + 0 \times 2^{0} + 1 \times 2^{-1} + 1 \times 2^{-2} = 26.75$$
Para distinguir los números con bases diferentes, se encierran los coeficientes entre paréntesis y se escribe un suscrito igual a la base usada. Un ejemplo de una base 5 sería:
$$(4021.2)_{5} = 4 \times 5^{3} + 0 \times 5^{2} + 2 \times 5^{1} + 1 \times 5^{0} + 2 \times 5^{-1} = (511.4)_{10}$$
Nótese que los únicos valores posibles para los coeficientes de base 5 pueden ser 0, 1, 2, 3 y 4.\\
En el sistema de numeración hexadecimal (base 16) se presentan los primeros diez dígitos del sistema decimal y las letras A, B, C, D, E y F para representar los números 
10, 11, 12, 13, 14 y 15. Por ejemplo, dado el número hexadecimal se obtiene su equivalente en decimal:
$$(B65F)_{16} = 11 \times 16^{3} + 6 \times 16^{2} + 5 \times 16^{1} + 15 \times 16^{0} = (46687)_{10}$$
En la Tabla \ref{table:sistemasNumeracion} se muestran los primeros 16 números en los sistemas decimal, binario, octal y hexadecimal.

  \begin{table}[ht]
	\centering	
	\begin{tabular}{c c c c}
	  \toprule
	  \textbf{Decimal (Base 10)} & \textbf{Binario (Base 2)} & \textbf{Octal (Base 8)} & \textbf{Hexadecimal (Base 16)} \\
	  \midrule
	  00 & 0000 & 00 & 0\\
	  01 & 0001 & 01 & 1\\
	  02 & 0010 & 02 & 2\\
	  03 & 0011 & 03 & 3\\
	  04 & 0100 & 04 & 4\\
	  05 & 0101 & 05 & 5\\
	  06 & 0110 & 06 & 6\\
	  07 & 0111 & 07 & 7\\
	  08 & 1000 & 10 & 8\\
	  09 & 1001 & 11 & 9\\
	  10 & 1010 & 12 & A\\
	  11 & 1011 & 13 & B\\
	  12 & 1100 & 14 & C\\
	  13 & 1101 & 15 & D\\
	  14 & 1110 & 16 & E\\
	  15 & 1111 & 17 & F\\
	\bottomrule
    \end{tabular}
    \caption{Equivalencias de los sistemas de numeración}
    \label{table:sistemasNumeracion}
  \end{table}   
  
\subsubsection{Conversión entre números de base diferente}

Un número binario puede ser convertido a decimal formando la suma de potencias de base 2, por ejemplo:
$$(1010.011)_{2} = 2^{3} + 2^{1} + 2^{-2} + 2^{-3} = (10375)_{10}$$
\textbf{Mediante este método puede convertirse un número de cualquier base a decimal, pero no viceversa.}

\subsection{Ejemplos de conversiones}

\begin{exampleT}{\rm
    Convertir $(41)_{10}$ a binario.\\
    La tabla \ref{table:Ejemplo1-1} muestra el desarrollo de la conversión.\\
    
   \begin{table}[!h]
   	\centering
	\begin{tabular}{c c}
	  \toprule
	    \textbf{Entero} & \textbf{Residuo} \\
	    \midrule
	    $41$ 	& \\
	    $41/2=20$	& 1\\
	    $20/2=10$	& 0\\
	    $10/2=5$	& 0\\
	    $5/2=2$	& 1\\
	    $2/2=1$	& 0\\
	    $1/2=0$	& 1\\
	  \bottomrule
	\end{tabular}
	\caption{Conversión del Ejemplo \ref{example:Ejemplo1-1}}
	\label{table:Ejemplo1-1}
   \end{table}

  Resultado: $(101001)_{2}$
  \label{example:Ejemplo1-1}
}\end{exampleT}
  

\begin{exampleT}{\rm
    Convertir $(153)_{10}$ a octal.\\
    La tabla \ref{table:Ejemplo1-2} muestra el desarrollo de la conversión.\\
    
    \begin{table}[!h]
    \centering
	\begin{tabular}{c c}
	  \toprule
	    \textbf{Entero} & \textbf{Residuo} \\
	    \midrule
	    $153$ 	& \\
	    $153/8=19$	& 1\\
	    $19/8=2$ 	& 3\\
	    $2/8=0$	& 2\\
	  \bottomrule
	\end{tabular}
	\caption{Conversión del Ejemplo \ref{example:Ejemplo1-2}}
	\label{table:Ejemplo1-2}
    \end{table}

  Resultado: $(231)_{8}$
  \label{example:Ejemplo1-2}
}\end{exampleT}

\begin{exampleT}{\rm
    Convertir $(0.6875)_{10}$ a binario.
    \begin{table}[!h]
    \centering
	\begin{tabular}{ccc}
	  \toprule
	     & \textbf{Entero} & \textbf{Residuo} \\
	     \midrule
	    $0.6875*2=1.375$ 	& $1$	& $0.375$\\
	    $0.375*2=0.75$	& $0$	& $0.75 $\\
	    $0.75*2=1.5$	& $1$	& $0.5$\\
	    $0.5*2=1.0$		& $1$	& $0$\\
	  \bottomrule
	\end{tabular}
	\caption{Conversión del Ejemplo \ref{example:Ejemplo1-3}}
	\label{table:Ejemplo1-3}
    \end{table}    
  Resultado: $(0.1011)_{2}$
  \label{example:Ejemplo1-3}
}\end{exampleT}

\begin{exampleT}{\rm
    Convertir $(0.513)_{10}$ a octal.
	\begin{table}[!h]
		\centering
    \begin{tabular}{ccc}
      \toprule
       & \textbf{Entero} & \textbf{Residuo} \\
       \midrule
	$0.513 * 8 = 4.104$ 	& $4$	& $0.104$\\
	$0.104 * 8 = 0.832$	& $0$	& $0.832$\\
	$0.832 * 8 = 6.656$	& $6$	& $0.656$\\
	$0.656 * 8 = 5.248$	& $5$	& $0.248$\\
	$0.248 * 8 = 1.984$ 	& $1$ 	& $0.984$\\
	$0.984 * 8 = 7.872$ 	& $7$ 	& $0.872$\\
      \bottomrule
    \end{tabular}  
    \caption{Conversión del Ejemplo \ref{example:Ejemplo1-4}}
    \label{table:Ejemplo1-4}
    	\end{table}
  \label{example:Ejemplo1-4}
}\end{exampleT}

  
\subsubsection{Conversión entre números de base diferente y no enteros}

  La conversión de números decimales con parte fraccionaria y entera se hace convirtiendo la parte fraccionaria y la entera por separado y posteriormente se combinan los dos resultados.
  Para la conversión de binario a octal y hexadecimal y viceversa, se considera que $2^{3} = 8$ y $2^{4} = 16$, por lo que cada dígito octal corresponde a tres dígitos binarios y cada 
  dígito hexadecimal corresponde a cuatro dígitos binarios.  


\begin{example}{\rm
    Convertir $(10110001101011.111100000110)_{2}$ a octal.
    $$10\, 110\, 001\, 101\, 011\, .\, 111\, 100\, 000\, 110)_{2} = (26153.7406)_{8}$$
}\end{example}

\begin{example}{\rm
    Convertir $(10110001101011.111100000110)_{2}$ a hexadecimal.
    $$(10\, 1100\, 0110\, 1011\, .\, 1111\, 0000\, 0110)_{2} = (2C6B.F06)_{16}$$

  La conversión de octal o hexadecimal a binario se hace por un proceso inverso al anterior. Cada dígito octal se convierte a un equivalente binario de tres dígitos, así como cada 
  dígito hexadecimal se convierte en un equivalente binario de cuatro dígitos.  
}\end{example}

\begin{example}{\rm
    Convertir $(673.124)_{8}$ a binario 
    $$(673.124)_{8} = (110\, 111\, 011\, .\, 001\, 010\, 100)_{2}$$
}\end{example}

\begin{example}{\rm
    Convertir $(306.D)_{16}$ a binario
    $$(306.D)_{16} = (0011\, 0000\, 0110\, .\, 1101)_{2}$$
}\end{example}


\begin{large}
\textbf{Ejercicios}
\end{large}

\begin{enumerate}
  \item Convierta el número decimal 250.5 a base 3, 4, 7, 8 y 16 respectivamente
  \item Convierta los siguientes números decimales a binarios: 12.0625, $10^{4}$, 673.23 y 1.998
  \item Convierta los siguientes números en base a las bases que se indican:
    \begin{enumerate}
      \item El decimal 225.225 a binario, octal y hexadecimal.
      \item El binario 11010111.110 a decimal, octal y hexadecimal.
      \item El octal 623.77 a decimal, binario y hexadecimal.
      \item El hexadecimal 2AC5.D a decimal, octal y binario.
    \end{enumerate}
    
  \item Convierta los siguientes números a decimal:
    \begin{enumerate}
      \item $(1001001.011)_{2}$
      \item $(12121)_{3}$
      \item $(1032.2)_{4}$
      \item $(4310)_{5}$
      \item $(0.342)_{6}$
      \item $(50)_{7}$
      \item $(8.3)_{9}$
      \item $(198)_{12}$
    \end{enumerate}
\end{enumerate}
  
  

\section{Representación computacional de números reales}

\section{Dígitos significativos}

\section{Redondeo y truncamiento}

\section{Algoritmos para la solución numérica de problemas}

\section{Plataformas y paquetes de simulación computacional}

\section{Bibliotecas de programas}