\chapter{Introducción}

%\section{Panorama histórico}

\section{Historia de los métodos numéricos}

\section{Razones de su aplicación en la Ingeniería}
La matemática y la física nos permite describir fenómenos de la naturaleza a través de modelos. El planteamiento de dichos modelos
generalmente resulta en un problema matemático. 

Por ejemplo, imagine que se desea calcular el área debajo de un puente y supongamos que el área formada por el puente no es una figura simple. 
Este cálculo podría hacerse a través de una integral, para ello es necesario obtener una función que modele la parte inferior del puente y plantear
la integral definida. Ahora al resolver la integral se podría obtener dicha área, y la calidad de la aproximación dependerá de lo preciso que sea
el modelo generado para el bajo-puente.

Por ello si se desea mejorar el cálculo habría que mejorar el modelo. Ahora si el modelo resulta una función complicada, ¿qué ocurriría si la integral
no se puede resolver analíticamente?. Debe ser claro que el área existe a pesar de que la integral pueda o no resolverse analíticamente. 

Para fines prácticos, nos interesa el resultado más no si éste se obtuvo de manera analítica o no. Es en este escenario donde los métodos numéricos
resultan muy útiles. 

\begin{verse}
\textit{Los métodos numéricos nos permiten obtener una aproximación a la solución de un problema matemático, más no resolverlo.} 
\end{verse}

¿Cuál es la diferencia entre resolver y aproximar a la solución de un problema matemático?. Resolver un problema matemático consiste en encontrar
una expresión, un teorema, o una regla para ese problema; es una solución que incluye una generalización para una familia de problemas semejantes.
Una aproximación a la solución es un valor numérico para un problema específico de esa familia.

El único precio que se paga por utilizar un método numérico es que las aproximaciones incluyen un error por su mera naturaleza.


\section{Conceptos básicos}

\begin{description}
	\item[Cifra significativa. ] El concepto de cifras o dígitos significativos se ha desarrollado para designar formalmente la confiabilidad de 
		un valor numérico. Las cifras significativas de un numero son aquellas que pueden utilizarse de forma confiable. Se trata del numero de dígitos 
		que se ofrecen con certeza, más uno estimado. Por ejemplo, el velocímetro y el odómetro de un automóvil mostrado en la figura.
	\item[Exactitud e incertidumbre]
	\item[Precisión y error]
\end{description}


\section{Tipos de errores}
Los métodos numéricos permiten encontrar aproximaciones a la solución de diversos problemas matemáticos. Sin embargo, el hecho de utilizar estos métodos introduce errores en la
aproximación. A continuación se describen algunos de los errores que ocurren.

\subsection{Error absoluto}

Una aproximación, por definición, es un valor cercano a la solución del problema, la cuál llamaremos absoluta. De esta forma, la diferencia entre la solución
absoluta y la aproximación a ella la llamaremos error absoluto. Es decir, si sumamos la aproximación y el error absoluto se obtiene la solución absoluta.\\

Este error absoluto es inevitable en los métodos numéricos y su estimación es compleja, incluso más que el método numérico que determina la aproximación. Por esta razón es que
el cálculo del error absoluto resulta impractico y se opta por utilizar estrategias en la obtención de la aproximación para minimizar este error.

\subsection{Error de redondeo y truncamiento}

Utilizar una herramienta digital para el cómputo numérico tiene muchas ventajas pero también desventajas. Entre sus ventajas más relevantes se encuentra el hecho de que los 
sistemas digitales permiten la realización de cálculos aritméticos rápidamente, lo cual es muy conveniente para los métodos iterativos. Una desventaja relevante de los sistemas
digitales es el hecho de que cuentan con una capacidad finita de memoria, aún y cuando esta capacidad de memoria se incrementa día a día, sigue siendo finita.\\ 

Tomemos un ejemplo específico, sea el número
\[ \dfrac{1}{3} = 0.33333\dots, \ \]
este número tiene un número infinito de cifras, un ligero cambio en alguna de esas cifras (por pequeña que sea) implica que el número ya no sea el mismo. Si el número no se escribe
en su forma racional, será necesario emplear el número infinito de cifras para realizar un trato \textit{preciso} del número. \\

Cuando este número se almacena en la memoria de un sistema digital, no podrán almacenarse todas las cifras de este número dado que la memoria es finita. Esto significa que el número 
será almacenado solamente con una cantidad limitada de cifras significativas, lo que significa que ya no será estrictamente el mismo número. Esto se llama \textbf{truncamiento}.
Por otro lado, cuando se realizan operaciones con números truncados podrían obtenerse resultados imprecisos, esto llama \textbf{error de truncamiento}.\\

Vayamos a un ejemplo. En este ejemplo utilizaremos nuevamente el número $\frac{1}{3}$ y realizaremos operaciones aritméticas con el para mostrar el error de truncamiento. Supongamos
que el sistema digital solo es capaz de almacenar 6 cifras significativas.

\begin{table}
	\centering
	\begin{tabular}{c|c}
	\textbf{Aritmética convencional} & \textbf{Aritmética digital con truncamiento}\\
	\hline
	$\dfrac{1}{3}$ & $0.333333$\\
	$3\times\dfrac{1}{3}$ & $3\times 0.333333$\\
	$1$ & $0.999999$		
	\end{tabular}
	\caption{Ejemplo del error de truncamiento}
	\label{table:errorTruncamiento}
\end{table}

En la tabla \ref{table:errorTruncamiento} se muestra la diferencia entre la aritmética convencional y digital, con el número completo y el número truncado. 
En el resultado es evidente que  formalmente $1\not= 0.999999$, y aunque se tenga un mayor número de cifras en el número digital siempre serán desiguales 
los resultados. En un caso práctico se dice que $1\approx 0.999999$, asumiendo que la diferencia es resultado del error de truncamiento.\\

El redondeo es similar al de truncamiento, solo que este ocurre cuando se ajusta el resultado de la operación a su inmediato superior o inferior, aquel que se
encuentre más cerca. El redondeo es una forma en la que los sistemas digitales pueden compensar el error de truncamiento, en el ejemplo mostrado en la tabla 
\ref{table:errorTruncamiento} lo compensaría apropiadamente, pero en otros casos no. Cuando el redondeo no compensa el truncamiento y en su lugar aleja al 
número de su valor absoluto, se dice que ocurre un \textbf{error de redondeo}. \\

Veamos un ejemplo en el que ocurre el error de redondeo y afecta al resultado obtienido en la operación. Considere los siguientes números y qu se hará
un redondeo de 6 cifras significativas del número.

\begin{table}[ht]
	\centering
	\begin{tabular}{c|c|c}
	\textbf{Número racional} & \textbf{Valor numérico} & \textbf{Valor numérico redondeado}\\
	\hline
	$2/7$ & $0.2857142857$ & $0.285710$\\
	$5/7$ & $0.7142857143$ & $0.714285$	
	\end{tabular}
	\caption{Valor numérico redondeado}
	\label{table:errorRedondeo}
\end{table}

Al realizar la suma de los números se obtiene un resultado inesperado e incorrecto debido al error de redondeo.

\begin{table}[ht]
	\centering
	\begin{tabular}{c|c}
	\textbf{Aritmética convencional} & \textbf{Aritmética digital con redondeo}\\
	\hline
	$\dfrac{2}{7}$ & $0.285710$\\
	+ & + \\
	$\dfrac{5}{7}$ & $0.714285$\\
	= & = \\
	$1$ & $0.999995$		
	\end{tabular}
	\caption{Ejemplo del error de redondeo}
	\label{table:errorTruncamiento}
\end{table}

Por último, cabe mencionar que estos errores pueden combinarse en las operaciones aritméticas digitales, la gravedad de sus efectos dependerán completamente 
de los números y las operaciones que se hagan con ellos.

\section{Uso de herramientas computacionales}
  





