\chapter{Sistemas de ecuaciones no lineales}

%\section{Solución de ecuaciones}
%\section{Búsqueda de valores iniciales Tabulación y graficación}
%\section{Métodos cerrados y sus interpretaciones geométricas (bisección y regla falsa)}
%\section{Iteración y convergencia de sistemas de ecuaciones}

\section{Método del Punto Fijo}
\subsection{Iteración funcional}

Un sistema de ecuaciones no lineales tiene la forma
\begin{equation}
	\begin{matrix}
		f_1(x_1,x_2,\dots,x_n) & = & 0,\\
		f_1(x_1,x_2,\dots,x_n) & = & 0,\\
		\vdots & & \vdots\\
		f_n(x_1,x_2,\dots,x_n) & = & 0,\\	
	\end{matrix}
	\label{eq:SNLpuntoFijo1} 	
\end{equation}
donde podemos considerar cada función $f_i$ como un mapeo de un vector $\textbf{x}=(x_1,x_2,\dots,x_n)^t$ del espacio $\mathbb{R}^n$ de dimensión
$n$ en la recta real $\mathbb{R}$. 
Este sistema de $n$ ecuaciones no lineales con $n$ incógnitas puede representarse también mediante la definición de una función $\mathbb{F}$ que
mapea $\mathbb{R}^n$ en $\mathbb{R}^n$ como
 
$$\mathbb{F}(x_1,x_2,\dots,x_n) = (f_1(x_1,x_2,\dots,x_n),f_2(x_1,x_2,\dots,x_n),\dots,f_n(x_1,x_2,\dots,x_n))^t.$$

Si se emplea una notación vectorial para representar las variables $x_1,x_2,\dots,x_n$, el sistema \ref{eq:SNLpuntoFijo1} adopta la forma
\begin{equation}
	\mathbb{F}(x) = 0
\end{equation}

Las funciones $f_a,f_2,\dots,f_n$ se denominan las \textbf{funciones coordenadas} de \textbf{F}.

\begin{exerciseT}
Aplique la iteración funcional para aproximar la solución con una exactitud de $10^{-5}$.

\begin{center}
  $\begin{array}{rcl}
    3x_1 - \cos (x_2x_3) - \frac{1}{2} & = & 0,\\
    x_1^2 - 81(x_2 + 0.1)^2 + \sin x_3 + 1.06 & = & 0,\\
    e^{-x_1x_2} + 20x_3 + \frac{10\pi - 3}{3} & = & 0
  \end{array}$
\end{center}

en la forma de punto fijo $x=G(x)$ resolviendo la í-esima ecuación para $x_i$, para demostrar que existe una única solución en

$$D=\{(x_1,x_2,x_3)^t | -a\leq x_i\leq 1, \mbox{ para cada } i=1,2,3\}.$$
e itere comenzando con $x^{(0)}=(0.1,0.1,-0.1)^t$ hasta obtener una precisión de $10^{-5}$ en la norma $l_\infty$.

\subsubsection*{Solución.}
Si resolvemos la í-esima ecuación para $x_i$, el sistema se transforma en un problema de punto fijo.

\begin{equation}
  \begin{array}{ccl}
    x_1 & = & \frac{1}{3}\cos (x_2x_3) + \frac{1}{6},\\
    x_2 & = & \frac{1}{9}\sqrt{x_1^2 + \sin x_3 + 1.06} - 0.1,\\ 
    x_3 & = & -\frac{1}{20}e^{-x_1x_2}-\frac{10\pi-3}{60}.
  \end{array}
  \label{eq:exSNLpuntoFijo1}
\end{equation}

Sea $G:\mathbb{R}^3\rightarrow\mathbb{R}^3$ definida por $G(x) = (g_1(x),g_2(x),g_3(x))^t$, donde

\begin{equation*}
  \begin{array}{ccl}
    g_1(x_1,x_2,x_3) & = & \frac{1}{3}\cos (x_2x_3) + \frac{1}{6},\\
    g_2(x_1,x_2,x_3) & = & \frac{1}{9}\sqrt{x_1^2 + \sin x_3 + 1.06} - 0.1,\\ 
    g_3(x_1,x_2,x_3) & = & -\frac{1}{20}e^{-x_1x_2}-\frac{10\pi-3}{60}.    
  \end{array}
\end{equation*}

Para aproximar el punto fijo \textbf{p}, escogemos $x^{(0)} = (0.1, 0.1, -0.1)^t$. La sucesión de vectores generada por

\begin{equation*}
  \begin{array}{ccl}
    x_1^{(k)} & = & \frac{1}{3}\cos \left(x_2^{(k-1)}x_3^{(k-1)}\right) + \frac{1}{6},\\
    x_2^{(k)} & = & \frac{1}{9}\sqrt{\left(x_1^{(k-1)}\right)^2 + \sin x_3^{(k-1)} + 1.06} - 0.1,\\ 
    x_3^{(k)} & = & -\frac{1}{20}e^{-x_1^{(k-1)}x_2^{(k-1)}}-\frac{10\pi-3}{60}.    
  \end{array}
\end{equation*}

converge en la solución única del sistema en \ref{eq:exSNLpuntoFijo1}. Los resultados en la tabla \ref{table:exSNLpuntoFijo1} se generan hasta que
$$||\textbf{x}^{(k)} - \textbf{x}^{(k-1)}||_\infty < 10^{-5}.$$

\begin{table}[H]
	\centering
	\begin{tabular}{ccccc}
		\toprule
		$k$ & $x_1^{(k)}$ & $x_2^{(k)}$ & $x_3^{(k)}$ & $||\textbf{x}^{(k)} - \textbf{x}^{(k-1)}||_\infty$ \\\midrule
		0 & 0.10000000 & 0.10000000 & -0.10000000 & \\
		1 & 0.49998333 & 0.00944115 & -0.52310127 & 0.423 \\
		2 & 0.49999593 & 0.00002557 & -0.52336331 & $9.4\times 10^{-3}$\\
		3 & 0.50000000 & 0.00001234 & -0.52359814 & $2.3\times 10^{-4}$\\
		4 & 0.50000000 & 0.00000003 & -0.52359847 & $1.2\times 10^{-5}$\\
		5 & 0.50000000 & 0.00000002 & -0.52359877 & $3.1\times 10^{-7}$\\
		\bottomrule
	\end{tabular}
	\caption{Iteraciones del Ejercicio \ref{ex:SNLpuntoFijo1}}
    \label{table:exSNLpuntoFijo1}
\end{table} 

Por lo tanto, la aproximación a la solución es $x^{(5)} = (0.50000000, 0.00000002, -0.52359877)$.  
 \label{ex:SNLpuntoFijo1}
\end{exerciseT}


\subsection{Iteración funcional acelerada}

Una forma de acelerar la convergencia de la iteración de punto fijo consiste en usar las estimaciones más recientes de $x_1^{(k)}, \dots, x_{i-1}^{(k)}$
para calcular $x_i^{(k)}$, igual que en el método de Gauss-Seidel para sistemas lineales. Entonces las ecuaciones componentes se transforman en

\begin{equation*}
  \begin{array}{ccl}
    x_1^{(k)} & = & \frac{1}{3}\cos \left(x_2^{(k-1)}x_3^{(k-1)}\right) + \frac{1}{6},\\
    x_2^{(k)} & = & \frac{1}{9}\sqrt{\left(x_1^{(k)}\right)^2 + \sin x_3^{(k-1)} + 1.06} - 0.1,\\ 
    x_3^{(k)} & = & -\frac{1}{20}e^{-x_1^{(k)}x_2^{(k)}}-\frac{10\pi-3}{60}.    
  \end{array}
\end{equation*}

Con $x^{(0)} = (0.1, 0.1, -0.1)^t$, los resultados de estos cálculos se dan en la tabla \ref{table:exSNLpuntoFijo2}.

\begin{table}[H]
	\centering
	\begin{tabular}{ccccc}
		\toprule
		$k$ & $x_1^{(k)}$ & $x_2^{(k)}$ & $x_3^{(k)}$ & $||\textbf{x}^{(k)} - \textbf{x}^{(k-1)}||_\infty$ \\\midrule
		0 & 0.10000000 & 0.10000000 & -0.10000000 & \\
		1 & 0.49998333 & 0.02222979 & -0.52304613 & 0.423 \\
		2 & 0.49997747 & 0.00002815 & -0.52359807 & $2.2\times 10^{-2}$\\
		3 & 0.50000000 & 0.00000004 & -0.52359877 & $2.8\times 10^{-5}$\\
		4 & 0.50000000 & 0.00000000 & -0.52359877 & $3.8\times 10^{-8}$\\
		\bottomrule
	\end{tabular}
    \caption{Convergencia Acelerada del Ejercicio \ref{ex:SNLpuntoFijo1}}
    \label{table:exSNLpuntoFijo2}
\end{table} 

Por lo tanto, la aproximación a la solución es $x^{(4)} = (0.50000000, 0.00000000, -0.52359877)$.    

\subsection*{Ejercicios}
Aplique la iteración funcional para aproximar la solución con una exactitud de $10^{-4}$.
\begin{enumerate}
	\item 
		$$G(x_1,x_2,x_3) = \left(\frac{\cos (x_2x_3) + 0.5}{3}, \frac{1}{25}\sqrt{x_1^2+0.3125}-0.03, -\frac{1}{20}e^{-x_1x_2}-\frac{10\pi-3}{60} \right)$$
		$$D=\{ (x_1,x_2,x_3)| -1\leq x_i\leq 1, i=1,2,3 \}$$
		
	\item
		$$G(x_1,x_2,x_3) = \left( \frac{13-x_2^2+4x_3}{15}, \frac{11+x_3-x_1^2}{10}, \frac{22+x_2^3}{25} \right)$$
		$$D=\{ (x_1,x_2,x_3)| 0\leq x_i\leq 1.5, i=1,2,3 \}$$
			
	\item 
		$$G(x_1,x_2,x_3) = \left(1-\cos(x_2x_3), 1-(1-x_1)^{1/4}-0.05x_3^2+0.15x_3, x_1^2+0.1x_2^2-0.01x_2+1 \right)$$
		$$D=\{ (x_1,x_2,x_3)| -0.1\leq x_1\leq 0.1, -0.1\leq x_2\leq 0.3, 0.5\leq x_3\leq 1.1 \}$$
			
	\item
		$$G(x_1,x_2,x_3) = \left( \frac{1}{3}\cos(x_2x_3) + \frac{1}{6}, -\frac{1}{9}\sqrt{x_1^2+\sin x_3+1.06}-0.1, -\frac{1}{20}e^{-x_1x_2}
			-\frac{10\pi -3}{60} \right)$$
		$$D=\{ (x_1,x_2,x_3)| -1\leq x_i\leq 1, i=1,2,3 \}$$			
\end{enumerate}


\section{Solución de sistemas de ecuaciones mediante Newton}

El problema del ejemplo \ref{ex:SNLpuntoFijo1} de la sección anterior convierte un problema de punto fijo convergente si se resuelve 
algebraicamente las tres ecuaciones para las tres variables $x_1,x_2$ y $x_3$. Sin embargo, no es usual poder encontrar una representación 
explícita para todas las variables. En esta sección estudiaremos un procedimiento algorítmico para efectuar la transformación en una 
situación más general.

Para construir el algoritmo que nos lleve a un método de punto fijo apropiado en el caso unidimensional, obtuvimos una función $\phi$ con 
la siguiente propiedad

\begin{equation*}
	g(x) = x - \phi(x)f(x)
\end{equation*}

da una convergencia cuadrática en el punto fijo $p$ de la función $g$. A partir de esta condición, el método de Newton se desarrolla 
seleccionando $\phi(x) = 1/f'(x)$, suponiendo que $f'(x)=0$.

La aplicación de un procedimiento semejante en el caso de dimensión $n$ incluye una matriz

\begin{equation}
	A = \begin{bmatrix}
		a_{11}(x) & a_{12}(x) & \cdots & a_{1n}(x) \\ 
		a_{21}(x) & a_{22}(x) & \cdots & a_{2n}(x) \\ 
		\vdots & \vdots &  & \vdots \\ 
		a_{n1}(x) & a_{n2}(x) & \cdots & a_{nn}(x)
	\end{bmatrix}
	\label{eq:SENLNewton1}
\end{equation}

donde todos los elementos $a_{ij}(x)$ son una función de $\mathbb{R}^n$ en $\mathbb{R}$. Esto requiere obtener $A(x)$ de modo que 
\begin{equation*}
	G(x) = x-A(x)^{-1}F(x)
\end{equation*}
de la convergencia cuadrática a la solución de $F(x) = 0$, suponiendo que $A(x)$ es no singular en el punto fijo $p$ de $G$.
\begin{theorem}[]
	Sea $p$ una solución de $G(x)=x$. Suponga que existe un número $\delta >0$ con
	\begin{enumerate}[(i)]
		\item $\delta g_i/\delta x_j$ es continua en $N_\delta = \{x| ||x-p||<\delta \}$, para cada $i=1,2,\dots,n$ y $j=1,2,\dots,n$;
		\item $\delta^2g_i(x)/(\delta x_j\delta x_k)$ es continua y $|\delta^2g_i(x)/(\delta x_j\delta x_k)|\leq M$ para alguna constante $M$, 
			siempre que $x\in N_{\delta}$, para cada $i=1,2,\dots, n$, $j=1,2,\dots,n$ y $k=1,2,\dots,n$; 
		\item $\delta g_i(p)/\delta x_k = 0$, para cada $i=1,2,\dots,n$ y $k=1,2,\dots,n$.
	\end{enumerate}		
	Entonces existe un número $\hat{\delta}\leq \delta$ tal que la sucesión generada por $x^{(k)} = G(x^{(k-1)})$ converge cuadráticamente a $p$ 
	para cualquier elección de $x^{(0)}$ a condición de que $||x^{(0)} - p|| < \hat{\delta}$. Más aún,
	\begin{equation*}
		||x^{(k)} - p||_\infty \leq \frac{n^2M}{2}||x^{(k-1)}-p||_\infty^2, \text{ para cada } k\geq 1.
	\end{equation*}	 
	\label{teo:SENLNewton}	
\end{theorem}

Para utilizar el teorema \ref{teo:SENLNewton} supongamos que $A(x)$ es una matriz de funciones $n\times n$ de $\mathbb{R}^n\rightarrow \mathbb{R}$ 
en la forma de la ecuación \ref{eq:SENLNewton1}, cuyas entradas específicas se escogerán más adelante. Supongamos además que $A(x)$ es no singular 
cerca de una solución $p$ de $F(x)=0$, y denotemos con $b_{ij}(x)$ la entrada de $A(x)^{-1}$ en el i-ésimo renglón y la j-ésima columna.

Para $G(x) = x - A(x)^{-1}F(x)$, tenemos $g_i(x) = x_i - \sum_{j=1}^n b_{ij}(x)f_j(x)$. Entonces 
\begin{equation*}
	\frac{\delta g_i}{\delta x_k}(x) = \left\{
		\begin{array}{lc}
			1 - \displaystyle{\sum_{j=1}^n \left( b_{ij}(x)\frac{\delta f_i}{\delta x_k}(x) + \frac{\delta b_{ij}}{\delta x_k}(x) f_j(x) \right)}, 
			& \mbox{ si } i=k,\\
			  & \\
			-\displaystyle{\sum_{j=1}^n}\left( b_{ij}(x)\frac{\delta f_i}{\delta x_k}(x) + \frac{\delta b_{ij}}{\delta x_k}(x) f_j(x) \right), 
			& \mbox{ si } i\not=k.
		\end{array}	
	\right.
\end{equation*}

El teorema \ref{teo:SENLNewton} implica que necesitamos $\dfrac{\delta g_i(p)}{\delta x_k} = 0$, para cada $i=1,2,\dots,n$ y $k=1,2,\dots,n$. 
Esto significa que, para $i=k$,
\begin{equation*}
  0 = 1 - \sum_{j=1}^n b_{ij}(p)\frac{\delta f_j}{\delta x_i}(p),
\end{equation*}

esto es,
\begin{equation}
	\sum_{j=1}^n b_{ij}(p)\dfrac{\delta f_j}{\delta x_i}(p) = 1
	\label{eq:SENLNewton2}
\end{equation}
	
Cuando $k\not=i$,
\[
	0 = -\sum_{j=1}^n b_{ij}(p) \dfrac{\delta f_j}{\delta x_k}(p) = 0,
\]

por lo que
\begin{equation}
	\sum_{j=1}^n b_{ij}(p)\dfrac{\delta f_j}{\delta x_k}(p) = 0.
	\label{eq:SENLNewton3}
\end{equation}

\subsection*{Matriz Jacobiana}

Al definir la matriz $J(x)$ por medio de
\[
	J(x) = 
	\left[
	\begin{matrix}
		\dfrac{\delta f_1}{\delta x_1}(x) & \dfrac{\delta f_1}{\delta x_2}(x) & \cdots & \dfrac{\delta f_1}{\delta x_n}(x) \\\\
		\dfrac{\delta f_2}{\delta x_1}(x) & \dfrac{\delta f_2}{\delta x_2}(x) & \cdots & \dfrac{\delta f_2}{\delta x_n}(x) \\
		\vdots & \vdots & & \vdots \\
		\dfrac{\delta f_n}{\delta x_1}(x) & \dfrac{\delta f_n}{\delta x_2}(x) & \cdots & \dfrac{\delta f_n}{\delta x_n}(x)
	\end{matrix}
	\right].
\]

Las condiciones \ref{eq:SENLNewton2} y \ref{eq:SENLNewton3} requieren que $A(p)^{-1}J(p) = I$, por lo que $A(p)=J(p)$. En
consecuencia, una elección apropiada de $A(x)$ es aquella que $A(x)=J(x)$, dado que cumple la condición $(iii)$ del teorema 
\ref{teo:SENLNewton}. La función $G$ definida por
\[
	G(x) = x - J(x)^{-1}F(x),
\]
y el procedimiento de iteración funcional es consecuencia de seleccionar $x^{(0)}$ y generar, para $k\geq 1$,
\[ 
	x^{(k)} = G\left(x^{(k-1)} \right) = x^{(k-1)} - J\left(x^{(k)}\right)^{-1}F\left(x^{(k-1)} \right).
\]

A esto se le llama \textbf{método de Newton para sistemas no lineales} y generalmente se espera que dé una convergencia 
cuadrática, siempre y cuando se conozca un valor inicial suficientemente preciso y exista $J(p)^{-1}$.

\begin{exerciseT}
Utilice el método de Newton para sistemas de ecuaciones no lineales para encontrar una aproximación a una solución. 
Considere una tolerancia $10^{-3}$.
\begin{center}
  $\begin{array}{rcl}
    3x_1 - \cos (x_2x_3) - \frac{1}{2} & = & 0,\\
    x_1^2 - 81(x_2 + 0.1)^2 + \sin x_3 + 1.06 & = & 0,\\
    e^{-x_1x_2} + 20x_3 + \frac{10\pi - 3}{3} & = & 0
  \end{array}$
\end{center}
Considere además que se sabe que la solución se encuentra en la región:
\[
	D=\{(x_1,x_2,x_3)^t | -a\leq x_i\leq 1, \mbox{ para cada } i=1,2,3\}.
\]

\subsection*{Planteamiento}
Se define la función $F(x)$:
\begin{align*}
	f_1(x_1,x_2,x_3) &= 3x_1 - \cos (x_2x_3) - \frac{1}{2} \\
	f_2(x_1,x_2,x_3) &= x_1^2 - 81(x_2 + 0.1)^2 + \sin x_3 + 1.06 \\
	f_3(x_1,x_2,x_3) &= e^{-x_1x_2} + 20x_3 + \frac{10\pi - 3}{3} \\
	F(x_1,x_2,x_3) &= \left(f_1(x_1,x_2,x_3), f_2(x_1,x_2,x_3), f_3(x_1,x_2,x_3) \right) 
\end{align*}

Se define la matriz Jacobiana $J(x)$:
\[
	J(x) = 
	\left[
	\begin{matrix}
		3 & x_3\sin x_2x_3 & x_2\sin x_2x_3 \\
		2x_1 & -162(x_2+0.1) & \cos x_3\\
		-x_2e^{-x_1x_2} & -x_1e^{-x_1x_2} & 20
	\end{matrix}
	\right].
\]
Se define el vector inicial $x^{(0)}$:
\[
	x^{(0)} = (0.1, 0.1, -0.1)
\]

\subsection*{Desarrollo}
Es necesario hacer las evaluaciones en la siguiente expresión 
\[ 
	x^{(k)} = x^{(k-1)} - J\left(x^{(k)}\right)^{-1}F\left(x^{(k-1)} \right)\\
	\forall k\geq 1
\]
comenzando con el vector inicial $x^{(0)}$. Cuando $k=1$, esta expresión se reduce a:
\begin{align*}
	x^{(1)} &= x^{(0)} - J\left(x^{(0)}\right)^{-1}F\left(x^{(0)} \right)\\
		&= (0.1, 0.1, -0.1) - J\left((0.1, 0.1, -0.1)\right)^{-1}F\left((0.1, 0.1, -0.1) \right)\\
		&= (0.4998696782, 0.01946684853, -0.5215204718)
\end{align*}
Este vector $x_1$ es la primer aproximación del método, se repite el procedimiento hasta alcanzar la tolerancia.
Los resultados se presentan en la tabla siguiente:
\begin{table}[H]
	\centering
	\begin{tabular}{cccc}
		\toprule
		$k$ & $x_1^{(k)}$ & $x_2^{(k)}$ & $x_3^{(k)}$ \\
		\midrule
		1 & 0.4998696782 & 0.0194668485 & -0.5215204718 \\
		2 & 0.5000142403 & 0.0015885914 & -0.5235569638 \\
		3 & 0.5000000113 & 0.0000124448 & -0.5235984500 \\
		4 & 0.5000000000 & $8.516\times 10^{-10}$ & -0.5235987755\\
		\bottomrule	
	\end{tabular}
\end{table}
En esta iteración se cumple la tolerancia:
\[ |x^{(4)} - x^{(3)}| = 1.244\times 10^{-5} < 10^{-3}\]

\subsection*{Resultado}
\begin{align*}
	x^{(4)} &= (0.5, 8.516\times 10^{-10}, -0.5235987755) \\
	ER &= 0.001718\%
\end{align*}

\end{exerciseT}


%\section{Método de Bairstow}
%\section{Aplicaciones}
