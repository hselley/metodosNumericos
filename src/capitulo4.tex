\chapter{Solución de sistemas de ecuaciones lineales}

\section{Métodos iterativos}

\subsection{Método de Jacobi}

En esta sección describiremos los métodos iterativos de Jacobi y Gauss-Seidel, métodos clásicos que datan de fines del siglo XVIII. Los métodos
iterativos rara vez se usan para resolver sistemas lineales de pequeña dimensión, ya que el tiempo necesario para conseguir una exactitud satisfactoria
rebasa el que requieren los métodos directos, como el  de eliminación gaussiana. Sin embargo, en el caso de sistemas grandes con un alto porcentaje de 
elementos cero, son eficientes tanto en almacenamiento de computadora como en el tiempo de cálculo. Este tipo de sistemas representan constantemente
en los análisis de circuitos y en la solución numérica de problemas con valores en la frontera y ecuaciones diferenciales parciales.

Un método iterativo con el cual se resuelve el sistema lineal $A\textbf{x}=\textbf{b}$ de dimensiones $n\times n$ comienza con una aproximación inicial $\textbf{x}^{(0)}$ a la
solución $\textbf{x}$ y genera una sucesión de vectores $\{\textbf{x}^{(k)}\}^\infty_{k=0}$ que convergen en $\textbf{x}$.

El método iterativo de Jacobi se obtiene resolviendo la i-ésima ecuación en $A\textbf{x}=\textbf{b}$ para $x_i$, lo que resulta en (siempre que $a_{ii}\not=0$)
\begin{equation*}
  x_i = \sum_{\substack{j=1\\j\not= i}}^n \left(-\frac{a_{ij}x_j}{a_{ii}} \right) + \frac{b_i}{a_{ii}}, \mbox{para } i=1,2,\dots ,n.
\end{equation*}

Para cada $k\geq 1$, se generan las componentes $x_i^{(k)}$ de $x^{(k)}$ a partir de las componentes de $x^{(k-1)}$ con
\begin{equation}
  x_i^{(k)} = \frac{1}{a_{ii}}\left[\sum_{\substack{j=1\\j\not= i}}\left( -a_{ij}x_j^{(k-1)}\right) +b_i\right], \mbox{ para } i=1,2,\dots,n.
  \label{eq:jacobi1}
\end{equation}

\begin{exerciseT}
El sistema lineal $A\textbf{x}=\textbf{b}$ dado por
\begin{center}
\begin{tabular}{lrcrcrcrcl}
$E_1$: & $10x_1$ & $-$ & $x_2$ & $+$ & $2x_3$ &  &  & $=$ & $6$,\\
$E_2$: & $-x_1$ & $+$ & $11x_2$ & $-$ & $x_3$ & $+$ & $3x_4$ & $=$ & $25$,\\
$E_3$: & $2x_1$ & $-$ & $x_2$ & $+$ & $10x_3$ & $-$ & $x_4$ & $=$ & $-11$,\\
$E_4$: &  &  & $3x_2$ & $-$ & $x_3$ & $+$ & $8x_4$ & $=$ & $15$\\
\end{tabular}
\end{center}
tiene la solución única $\textbf{x}=(1,2,-1,1)^t$. Utilice el método iterativo de Jacobi para encontrar aproximaciones $\textbf{x}^{(k)}$ a $\textbf{x}$
comenzando con $\textbf{x}^{(0)} = (0,0,0,0)^t$ hasta
\begin{equation*}
 ||\textbf{x}^{(k)} - \textbf{x}^{(k-1)}||_{\infty} < 10^{-2}.
\end{equation*}

Primero debemos despejar $x_i$ de la ecuación $E_i$ para cada $i=1,2,3,4$ para obtener
\begin{table}[H]
	\centering
	\begin{tabular}{lcrcrcrcrcr}
		$x_1$ & $=$ &  &  & $\frac{1}{10}x_2$ & $-$ & $\frac{1}{5}x_3$ & & & $+$ & $\frac{3}{5}$,\\
		$x_2$ & $=$ & $\frac{1}{11}x_1$ &  &  & $+$ & $\frac{1}{11}x_3$ & $-$ & $\frac{3}{11}x_4$ & $+$ & $\frac{25}{11}$,\\
		$x_3$ & $=$ & $-\frac{1}{5}x_1$ & $+$ & $\frac{1}{10}x_2$ & & & $+$ & $\frac{1}{10}x_4$ & $-$ & $\frac{11}{10}$,\\
		$x_4$ & $=$ &  & $-$ & $\frac{3}{8}x_2$ & $+$ & $\frac{1}{8}x_3$ & & & $+$ & $\frac{15}{8}$,\\
	\end{tabular}
\end{table}

a partir de la aproximación inicial $\textbf{x}^{(0)}=(0,0,0,0)^t$ tenemos que $\textbf{x}^{(1)}$ está dada por
\begin{table}[H]
	\centering
	\begin{tabular}{lcrcrcrcrcrcr}
	$x_1^{(1)}$ & $=$ &  &  & $\frac{1}{10}x_2^{(0)}$ & $-$ & $\frac{1}{5}x_3^{(0)}$ & & & $+$ & $\frac{3}{5}$ & $=$ & $0.6000$,\\
	$x_2^{(1)}$ & $=$ & $\frac{1}{11}x_1^{(0)}$ &  &  & $+$ & $\frac{1}{11}x_3^{(0)}$ & $-$ & $\frac{3}{11}x_4^{(0)}$ & $+$ & $\frac{25}{11}$ & $=$ 
	&  $2.2727$,\\
	$x_3^{(1)}$ & $=$ & $-\frac{1}{5}x_1^{(0)}$ & $+$ & $\frac{1}{10}x_2^{(0)}$ & & & $+$ & $\frac{1}{10}x_4^{(0)}$ & $-$ & $\frac{11}{10}$ & $=$ & $-1.100$,\\
	$x_4^{(1)}$ & $=$ &  & $-$ & $\frac{3}{8}x_2^{(0)}$ & $+$ & $\frac{1}{8}x_3^{(0)}$ & & & $+$ & $\frac{15}{8}$ & $=$ & $1.875$,\\
\end{tabular}
\end{table}

Las siguientes iteraciones $\textbf{x}^{(k)}=(x_1^{(k)}, x_2^{(k)}, x_3^{(k)}, x_4^{(k)})^t$, se generan de manera semejante y se producen los valores 
que se muestran en la tabla \ref{table:jacobi1}.

\begin{table}[H]
	\centering
	\begin{tabular}{r|rrrrrrrrr}
		\toprule
		$k$ & 0 & 1 & 2 & 3 & 4 & 5 & 6 & 7 & 8\\\midrule
		$x_1^{(k)}$ & 0.0000 & 0.6000 & 1.0473 & 0.9326 & 1.0152 & 0.9890 & 1.0032 & 0.9981 & 1.0006 \\
		$x_2^{(k)}$ & 0.0000 & 2.2727 & 1.7159 & 2.0530 & 1.9537 & 2.0114 & 1.9922 & 2.0023 & 1.9987 \\
		$x_3^{(k)}$ & 0.0000 &-1.1000 &-0.8052 &-1.0493 &-0.9681 &-1.0103 &-0.9945 &-1.0020 &-0.9990 \\
		$x_4^{(k)}$ & 0.0000 & 1.8750 & 0.8852 & 1.1309 & 0.9739 & 1.0214 & 0.9944 & 1.0036 & 0.9989 \\
		\bottomrule
      \end{tabular}
	\caption{Iteraciones del Ejercicio \ref{ex:jacobi1}}
	\label{table:jacobi1}
\end{table}  
\label{ex:jacobi1}
El método termina debido a que ya se ha cumplido con la tolerancia $10^{-2}$ para los valores obtenidos en la última iteración.
$$||\textbf{x}^{(8)}-\textbf{x}^{(7)}|| = 0.00708715 < 10^{-2}$$
\end{exerciseT}

En el algoritmo \ref{algo:jacobi} se implementa el método iterativo de Jacobi. El método resuelve el sistema $A\textbf{x}=\textbf{b}$ dada 
una aproximación inicial $\textbf{x}^{(0)}$.

\begin{algorithm}[ht]
  \SetKwInOut{Input}{Entrada}
  \SetKwInOut{Output}{Salida}
  \SetKwFunction{Salida}{Salida}
  \Input{$n, A = [a_{ij}]$, donde $1\leq i,j\leq n$, $b=[b_i]$ donde $1\leq i\leq n$, $XO=x^{(0)}$,TOL y $N$}
  \Output{$x_1,x_2,\dots,x_n$}
  \Begin{
    $k=1$\\
    \While{$k\leq N$}{
      \For{$i=1$ \KwTo $n$}{
	$\displaystyle{x_i = \frac{1}{a_{ii}}\left[-\sum_{\substack{j=1\\j\not=i}}^n (a_{ij}XO_j) + b_i\right]}$
      }
      \If{$||\textbf{x}-\textbf{XO}||<TOL$}{
	\Salida{$x_1,x_2,\dots,x_n$}\\
	$fin$
      }
      $k=k+1$\\
      \For{$i=1$ \KwTo $n$}{
	$XO_i = x_i$
      } 
    }
    \Salida{Número máximo de iteraciones excedido.}
  }
\caption{Método iterativo de Jacobi}
\label{algo:jacobi}
\end{algorithm}

\subsection*{Ejercicios}
  \begin{enumerate}
    \item Obtenga las dos primeras iteraciones del método de Jacobi para los siguientes sistemas lineales, use $\textbf{x}^{(0)} = \textbf{0}$.
      \begin{enumerate}
		\begin{multicols}{2}
		\item $\begin{array}{r}
				3x_1 - x_2 + x_3 = 1\\
				3x_1 + 6x_2 + 2x_3 = 0\\
				3x_1 + 3x_2 + 7x_3 = 4\\
			  \end{array}$
		\item $\begin{array}{r}
				10x_1 - x_2 = 9\\
				-x_1 + 10x_2 - 2x_3 = 7\\
				-2x_2 + 10x_3 = 6\\
			\end{array}$
		\item $\begin{array}{r}
				10x_1 + 5x_2 = 6\\
				5x_1 + 10x_2 - 4x_3 = 25\\
				-4x_2 + 8x_3 - x_4 = -11\\
				-x_3 + 5x_4 = -11\\
			   \end{array}$
		\item $\begin{array}{r}
				4x_1 + x_2 + x_3 + x_5 = 6\\
				-x_1 - 3x_2 + x_3 + x_4 = 6\\
				2x_1 + x_2 + 5x_3 - x_4 - x_5 = 6\\
				-x_1 - x_2 - x_3 + 4x_4 = 6\\
				2x_2 - x_3 + x_4 + 4x_5 = 6
			\end{array}$
		\end{multicols}
      \end{enumerate}		
    \item Obtenga las dos primeras iteraciones del método de Jacobi para los siguientes sistemas lineales, use $\textbf{x}^{(0)} = \textbf{0}$.
      \begin{enumerate}
		\begin{multicols}{2}
		\item $\begin{array}{r}
				4x_1 + x_2 - x_3 = 5\\
				-x_1 + 3x_2 + x_3 = -4\\
				2x_1 + 2x_2 + 5x_3 = 1
			  \end{array}$
		\item $\begin{array}{r}
				-2x_1 + x_2 + \frac{1}{2}x_3 = 4\\
				x_1 - 2x_2 - \frac{1}{2}x_3 = -4\\
				x_2 + 2x_3 = 0
			\end{array}$
		\item $\begin{array}{r}
				4x_1 + x_2 - x_3 + x_4 = -2\\
				x_1 + 4x_2 - x_3 - x_4 = -1\\
				-x_1 - x_2 + 5x_3 + x_4 = 0\\
				x_1 - x_2 + x_3 + 3x_4 = 1
			   \end{array}$
		\item $\begin{array}{r}
				4x_1 - x_2 - x_4 = 0\\
				-x_1 + 4x_2 - x_3 - x_5 = 5\\
				-x_2 + 4x_3 - x_6 = 0\\
				-x_1 + 4x_4 - x_5 = 6\\
				-x_2 - x_4 + 4x_5 - x_6 = -2\\
				-x_3 - x_5 + 4x_6 = 6
			\end{array}$
		\end{multicols}		
      \end{enumerate}
\end{enumerate}



\subsection{Método de Gauss-Seidel}
Si se analiza la expresión \ref{eq:jacobi1} se puede observar una posible mejora del algoritmo \ref{algo:jacobi}. Utilizamos las componentes
de $\textbf{x}^{(k-1)}$ para calcular todas las componentes $x_i^{(k)}$ de $\textbf{x}^{(k)}$. Pero para $i>1$, ya se calcularon los componentes
$x_1^{(k)}, \dots, x_{i-1}^{(k)}$ de $\textbf{x}^{(k)}$, y probablemente sean mejores aproximaciones a las soluciones reales $x_1,\dots,x_{i-1}$
que $x_1^{(k-1)},\dots, x_{i-1}^{(k-1)}$. Entonces parece más razonable calcular $x_i^{(k)}$ por medio de los valores calculados más recientemente.
Esto es, podemos usar
\begin{equation}
  x_i^{(k)}=\frac{1}{a_{ii}}\left[-\sum_{j=1}^{i-1}\left(a_{ij}x_j^{(k)}\right) - \sum_{j=i+1}^{n}\left(a_{ij}x_j^{(k-1)}\right) + b_i \right],
  \label{eq:Gauss-Seidel}
\end{equation}
para cada $i=1,2,\dots,n$, en vez de la ecuación \ref{eq:jacobi1}. A esta modificación se le llama \textbf{método iterativo de Gauss-Seidel}.

\begin{exerciseT}
Utilice el método iterativo Gauss-Seidel para encontrar las soluciones aproximadas de
$$\begin{array}{r}
    10x_1 - x_2 + 2x_3 = 6,\\
    -x_1 + 11x_2 - x_3 + 3x_4 = 25,\\
    2x_1 - x_2 + 10x_3 - x_4 = -11,\\
    3x_2 - x_3 + 8x_4 = 15
  \end{array}$$
comenzando con $\textbf{x}=(0,0,0,0)^t$ e iterando hasta $10^{-2}$.

\subsubsection*{Solución.}
En primer lugar se escribe el sistema de la siguiente forma:
\begin{table}[H]
	\centering
	\begin{tabular}{lcrcrcrcrcr}
    	$x_1^{(k)}$ & $=$ &  &  & $\frac{1}{10}x_2^{(k-1)}$ & $-$ & $\frac{1}{5}x_3^{(k-1)}$ & & & $+$ & $\frac{3}{5}$,\\
	    $x_2^{(k)}$ & $=$ & $\frac{1}{11}x_1^{(k)}$ &  &  & $+$ & $\frac{1}{11}x_3^{(k-1)}$ & $-$ & $\frac{3}{11}x_4^{(k-1)}$ & $+$ & $\frac{25}{11}$,\\
    	$x_3^{(k)}$ & $=$ & $-\frac{1}{5}x_1^{(k)}$ & $+$ & $\frac{1}{10}x_2^{(k)}$ & & & $+$ & $\frac{1}{10}x_4^{(k-1)}$ & $-$ & $\frac{11}{10}$,\\
	    $x_4^{(k)}$ & $=$ &  & $-$ & $\frac{3}{8}x_2^{(k)}$ & $+$ & $\frac{1}{8}x_3^{(k)}$ & & & $+$ & $\frac{15}{8}$.\\
  	\end{tabular}
\end{table}

La sustitución ocurre de la siguiente forma:
\begin{table}[H]
	\centering
	\begin{tabular}{lcrcrcrcrcrcr}
    	$x_1^{(k)}$ & $=$ &  &  & $\frac{1}{10}(0)$ & $-$ & $\frac{1}{5}(0)$ & & & $+$ & $\frac{3}{5}$ & $=$ & $\frac{3}{5} = 0.6000$,\\
    	$x_2^{(k)}$ & $=$ & $\frac{1}{11}(0.6)$ &  &  & $+$ & $\frac{1}{11}(0)$ & $-$ & $\frac{3}{11}(0)$ & $+$ & $\frac{25}{11}$ & $=$ 
    	& $\frac{128}{55}=2.3272$,\\
	    $x_3^{(k)}$ & $=$ & $-\frac{1}{5}(\frac{3}{5})$ & $+$ & $\frac{1}{10}(\frac{128}{55})$ & & & $+$ & $\frac{1}{10}(0)$ & $-$ 
	    & $\frac{11}{10}$ & $=$ & $-\frac{543}{550} = -0.9873$,\\
	    $x_4^{(k)}$ & $=$ &  & $-$ & $\frac{3}{8}(\frac{128}{55})$ & $+$ & $\frac{1}{8}(-\frac{543}{550})$ & & & $+$ & $\frac{15}{8}$ & $=$ 
	    & $\frac{3867}{4400} = 0.8789$.\\
  \end{tabular}
\end{table}

Por tanto cuando $\textbf{x}^{(0)} = (0,0,0,0)^t$ tenemos $x^{(1)} = (0.6000, 2.3272, -0.9873, 0.8789)^t$. Las iteraciones subsecuentes generan los valores
de la tabla \ref{table:gauss-seidel1}.

\begin{table}[H]
	\centering
	\begin{tabular}{r|rrrrr}
		\toprule
		$k$ & 0 & 1 & 2 & 3 & 4 \\\midrule
		$x_1^{(k)}$ & 0.0000 & 0.6000 & 1.030 & 1.0065 & 1.0009 \\
		$x_2^{(k)}$ & 0.0000 & 2.2727 & 2.037 & 2.0036 & 2.0003 \\
		$x_3^{(k)}$ & 0.0000 &-0.9873 &-1.014 &-1.0025 &-1.0003 \\
		$x_4^{(k)}$ & 0.0000 & 0.8789 &0.9844 & 0.9983 & 0.9999 \\
		\bottomrule
	\end{tabular}
	\caption{Iteraciones del Ejercicio \ref{ex:gauss-seidel1}}
    \label{table:gauss-seidel1}
\end{table}  

Dado que $||\textbf{x}^{(4)} - \textbf{x}^{(3)}|| = 0.00710962 < 10^{-2}$ se acepta $\textbf{x}^{(4)}$ como una aproximación razonable a la solución.  
\label{ex:gauss-seidel1}
\end{exerciseT}

El algoritmo \ref{algo:gauss-seidel} describe el método Gauss-Seidel.

\begin{algorithm}[ht]
  \SetKwInOut{Input}{Entrada}
  \SetKwInOut{Output}{Salida}
  \SetKwFunction{Salida}{Salida}
  \Input{$n, A = [a_{ij}]$, donde $1\leq i,j\leq n$, $b=[b_i]$ donde $1\leq i\leq n$, $XO=x^{(0)}$,TOL y $N$}
  \Output{$x_1,x_2,\dots,x_n$}
  \Begin{
    $k=1$\\
    \While{$k\leq N$}{
      \For{$i=1$ \KwTo $n$}{
	$\displaystyle{x_i = \frac{1}{a_{ii}}\left[-\sum_{j=1}^{i-1} a_{ij}x_j -\sum_{j=i+1}^{n}a_{ij}XO_j + b_i\right]}$
      }
      \If{$||\textbf{x}-\textbf{XO}||<TOL$}{
	\Salida{$x_1,x_2,\dots,x_n$}\\
	$fin$
      }
      $k=k+1$\\
      \For{$i=1$ \KwTo $n$}{
	$XO_i = x_i$
      } 
    }
    \Salida{Número máximo de iteraciones excedido.}
  }
\caption{Método iterativo de Gauss-Seidel}
\label{algo:gauss-seidel}
\end{algorithm}

Los resultados de los ejemplos \ref{ex:jacobi1} y \ref{ex:gauss-seidel1} parecen indicar que este método es superior al de Jacobi, y generalmente es 
así. Sin embargo hay sistemas lineales en los que el método de Jacobi converge y el de Gauss-Seidel no.

\subsection*{Ejercicios}
  \begin{enumerate}
    \item Obtenga las dos primeras iteraciones del método de Gauss-Seidel para los siguientes sistemas lineales, use $\textbf{x}^{(0)} = \textbf{0}$.
      \begin{enumerate}
		\begin{multicols}{2}
		\item $\begin{array}{r}
				3x_1 - x_2 + x_3 = 1\\
				3x_1 + 6x_2 + 2x_3 = 0\\
				3x_1 + 3x_2 + 7x_3 = 4\\
			  \end{array}$
		\item $\begin{array}{r}
				10x_1 - x_2 = 9\\
				-x_1 + 10x_2 - 2x_3 = 7\\
				-2x_2 + 10x_3 = 6\\
			\end{array}$
		\item $\begin{array}{r}
				10x_1 + 5x_2 = 6\\
				5x_1 + 10x_2 - 4x_3 = 25\\
				-4x_2 + 8x_3 - x_4 = -11\\
				-x_3 + 5x_4 = -11\\
			   \end{array}$
		\item $\begin{array}{r}
				4x_1 + x_2 + x_3 + x_5 = 6\\
				-x_1 - 3x_2 + x_3 + x_4 = 6\\
				2x_1 + x_2 + 5x_3 - x_4 - x_5 = 6\\
				-x_1 - x_2 - x_3 + 4x_4 = 6\\
				2x_2 - x_3 + x_4 + 4x_5 = 6
			\end{array}$
		\end{multicols}
      \end{enumerate}		
    \item Obtenga las dos primeras iteraciones del método de Gauss-Seidel para los siguientes sistemas lineales, use $\textbf{x}^{(0)} = \textbf{0}$.
      \begin{enumerate}
		\begin{multicols}{2}
		\item $\begin{array}{r}
				4x_1 + x_2 - x_3 = 5\\
				-x_1 + 3x_2 + x_3 = -4\\
				2x_1 + 2x_2 + 5x_3 = 1
			  \end{array}$
		\item $\begin{array}{r}
				-2x_1 + x_2 + \frac{1}{2}x_3 = 4\\
				x_1 - 2x_2 - \frac{1}{2}x_3 = -4\\
				x_2 + 2x_3 = 0
			\end{array}$
		\item $\begin{array}{r}
				4x_1 + x_2 - x_3 + x_4 = -2\\
				x_1 + 4x_2 - x_3 - x_4 = -1\\
				-x_1 - x_2 + 5x_3 + x_4 = 0\\
				x_1 - x_2 + x_3 + 3x_4 = 1
			   \end{array}$
		\item $\begin{array}{r}
				4x_1 - x_2 - x_4 = 0\\
				-x_1 + 4x_2 - x_3 - x_5 = 5\\
				-x_2 + 4x_3 - x_6 = 0\\
				-x_1 + 4x_4 - x_5 = 6\\
				-x_2 - x_4 + 4x_5 - x_6 = -2\\
				-x_3 - x_5 + 4x_6 = 6
			\end{array}$
		\end{multicols}		
      \end{enumerate}
\end{enumerate}
